\documentclass[a4paper, 11pt, oneside, polutonikogreek, italian]{article}
\usepackage{lmodern}
\usepackage[T1]{fontenc}
\usepackage{wasysym}
\usepackage{tipa}
% Load encoding definitions (after font package)
\usepackage{svg}
\usepackage{tablefootnote}
\usepackage{listings}
\lstset{basicstyle=\ttfamily}
\usepackage{ipa}
\usepackage{tabularx}
% Babel package:
\usepackage[italian]{babel}
\usepackage{longtable}
% With XeTeX$\$LuaTeX, load fontspec after babel to use Unicode
% fonts for Latin script and LGR for Greek:
\ifdefined\luatexversion \usepackage{fontspec}\fi
\ifdefined\XeTeXrevision \usepackage{fontspec}\fi

% ```Lipsiakos"' italic font `cbleipzig`:
\newcommand*{\lishape}{\fontencoding{LGR}\fontfamily{cmr}%
		 \fontshape{li}\selectfont}
\DeclareTextFontCommand{\textli}{\lishape}
\usepackage{booktabs}
\setlength{\emergencystretch}{15pt}
\usepackage{microtype}
\usepackage{float}
\usepackage{graphicx}
\graphicspath{ {./ } }
\usepackage[figurename=]{caption}
%define custom symbols
\newcommand*\svgaaa{\includesvg[height=0.6em]{svgs/001.svg}}
\newcommand*\svgaab{\raisebox{-0.7ex}{\includesvg[height=0.7em]{svgs/002.svg}}}
\newcommand*\svgaac{\includesvg[height=0.6em]{svgs/008.svg}}
\newcommand*\svgaad{\includesvg[height=0.6em]{svgs/009.svg}}
\newcommand*\svgaae{\includesvg[height=0.6em]{svgs/010.svg}}
\newcommand*\svgaaf{\includesvg[height=0.6em]{svgs/011.svg}}
\newcommand*\svgaag{\includesvg[height=0.6em]{svgs/012.svg}}
\newcommand*\svgaah{\includesvg[height=0.6em]{svgs/013.svg}}
\newcommand*\svgaai{\includesvg[height=0.6em]{svgs/014.svg}}
\newcommand*\svgaaj{\includesvg[height=0.6em]{svgs/015.svg}}
\newcommand*\svgaak{\includesvg[height=0.6em]{svgs/016.svg}}
\newcommand*\svgaal{\includesvg[height=0.6em]{svgs/017.svg}}
\newcommand*\svgaam{\includesvg[height=0.6em]{svgs/018.svg}}
\newcommand*\svgaan{\includesvg[width=0.7em]{svgs/019.svg}}
\newcommand*\svgaao{\includesvg[width=0.8em]{svgs/020.svg}}
\newcommand*\svgaap{\includesvg[width=0.8em]{svgs/021.svg}}
\newcommand*\svgaaq{\includesvg[width=0.8em]{svgs/022.svg}}
\newcommand*\svgaar{\includesvg[height=0.6em]{svgs/023.svg}}
\newcommand*\svgaas{\includesvg[height=0.7em]{svgs/024.svg}}
\newcommand*\svgaat{\includesvg[height=0.6em]{svgs/025.svg}}
\newcommand*\svgaau{\includesvg[height=0.6em]{svgs/026.svg}}

\usepackage{tocloft}
\cftsetindents{subsubsection}{5em}{5em}
\cftsetindents{subsection}{3em}{5em}
\cftsetindents{section}{0em}{5em}

\begin{document}
\begin{titlepage}
    \centering                     % Center everything on the page

    % -------------------------------------------------
    % Title rules
    % -------------------------------------------------
    \rule{\textwidth}{1.6pt}\\[-\baselineskip]\vskip2pt
    \rule{\textwidth}{0.4pt}\\[1\baselineskip]

    % -------------------------------------------------
    % Title
    % -------------------------------------------------
    {\scshape\Huge Mappæ Clavicula}\\[0.2\baselineskip]

    % -------------------------------------------------
    % Bottom title rules
    % -------------------------------------------------
    \rule{\textwidth}{0.4pt}\\[-\baselineskip]\vskip3.2pt
    \rule{\textwidth}{1.6pt}\\[1\baselineskip]

    % -------------------------------------------------
    % Subtitle (long Latin sentence)
    % -------------------------------------------------
    {\scshape Letter from \\\Large Sir Thomas Phillipps\\\small Bart., F. R. S., F. S. A.}

    \vspace{\baselineskip}

    {\scshape Extract from\\\emph{Archaeologia: or Miscellaneous Tracts relating to Antiquity}\\ Vol. 32}

    % -------------------------------------------------
    % Date / defense statement / author line
    % -------------------------------------------------


    % -------------------------------------------------
    % Place / publisher info (pushed to bottom)
    % -------------------------------------------------
    \vfill

    {\scshape London, 1847\\\small Society of Antiquaries of London}
 
    % -------------------------------------------------
    % Edition / license line
    % -------------------------------------------------
    {\scshape Solar Anamnesis Edition}\\[0.2\baselineskip]
    {\scshape\small CC0 1.0 Universal}
\end{titlepage}
\setlength{\parskip}{1mm plus1mm minus1mm}
\clearpage
\tableofcontents
\clearpage
\section*{}
\begin{center}
\emph{Addressed to Albert Way, Esq., Director, communicating a transcript of a MS. Treatise on the preparation of Pigments, and on various processes of the Decorative Arts practised during the Middle Ages, written in the twelfth century, and entitled Mappæ Clavicula.}
\end{center}
\begin{center}
Read 22nd January, 1846.

Middle Hill, 9th January, 1846.
\end{center}
\paragraph{}
My Dear Sir,

The Manuscript entitled \textbf{Mappæ Clavicula}, signifying the Little Key of Drawing, or Painting, is a small duodecimo volume of sixty-seven leaves of vellum, written in the twelfth century. It appears to be perfect, except a leaf torn out between pp. 64 and 65 of the modern paging, and a little cropping in two leaves.

It was purchased at Paris, in 1824, from the Rev. Mons. Allard, Curé of the church of Saint Eustache, and had previously belonged to Mons. J. Rabaut.

I have always considered it to be one of the most curious and interesting treatises relating to the art of painting, or rather the composition of colours for painting, in existence. It is very remarkable in many respects. Its antiquity exceeds that of all other works treating of similar arts with which I am acquainted. The character of the writing appears to be that used in England or Flanders in the time of Henry 2nd; but from an English word being used in the work the presumption will be, of the two nations, in favour of England. Should it be an English work, we may fairly infer that these ``secrets'' are those which were known to our Saxon artists and artificers. The passage which quotes the English word is in chapter 190. where the shrub ``caprifolium'' is translated ``goat tree.''\footnote{The term ``gate'' for ``goat'' explains the reason why the family of Yate bear three gates for their arms, and a goat's head for their crest; a gate formerly being pronounced (as it is now in some counties) a yat, or yate: both are, in fact, puns upon the name.} This is a singular circumstance, and seems to me to indicate, as I said before, that the author or the transcriber was an Englishman, for had he been of any other nation he would most naturally have translated it by the language of his own country. Moreover, in the very next chapter, he mentions the herb ``\emph{greningpert},'' a corruption, I suspect, of ``\emph{greningwert},'' the Saxon \textwynn\hspace{0.4mm} being easily mistaken for a \emph{p}; (and we know that several English names of herbs end in \emph{wort},\footnote{Wurt in old German and Saxon signifies a herb, as we know from the Latin name of Wurtzburgh, which is translated Herbipolis.} as St. John's Wort, etc.) which I consider an additional mark of his being an English author. It is evidently the work of a person well practised in the arts which he describes, and he has apparently some classical learning, since there is little doubt that he had read Vitruvius, for he gives an extract from him, although he does not mention him by name; the chapter on Brick Walls, ``De latericiis Parietibus,'' being, I believe, taken from that author, and probably also the chapter on Lime and Sand.

The author does not confine himself to the composition of colours, but treats of a variety of other subjects, in a concise and simple manner, and generally very intelligibly; as for instance, Architecture, Mensuration of Altitudes, the Art of War, etc. The various articles on these and other subjects may be summed up in the following: namely, 1. Softening gold so as to mould it with the fingers; 2. To melt gold without the aid of fire; 3. To write in gold, silver, and quicksilver letters; 4. To gild other metals; 5. To restore tarnished silver to its brilliancy; 6. To gild on stone, wood, glass, and leather; 7. To proportion buildings; 8. To build in water; 9. To build brick walls; 10. The choice of sand; 11. To make mortar; 12. To make glue for wood or bone; 13. To make glass; 14. To stain glass with various colours; 15. To stain or dye leather, bone, horn, and wood; 16. To varnish paintings; 17. To prepare size for painting on walls; 18. To make soap and starch.

He gives ns instructions in other arts, which have been long lost; such as, 1. The composition of Niello; 2. Encaustic; 3. Ancient ink; 4. The mode of softening glass, crystal, and ivory; 5. Of making fiery arrows, and empoisoning them; 6. The mode of using the battering-ram, etc.

The variety of synonyms given to the same trees, shrubs, or herbs in pp. 238 and 239 will be interesting to the botanist, and valuable to the artist, for by some one of these different names he may be able to identify a plant which would otherwise remain unknown to him. One secret the author appears to be afraid of entrusting even to his own son, or, at all events, not to his book, for he conceals the names of the materials under fictitious letters, as in chapter 212. where he speaks of the composition of a water which may be set on fire and yet the material which it surrounds shall remain unconsumed. His mode of measuring heights is simple and easy: the method of making fiery and empoisoned arrows, and of using, preserving, and improving the battering-ram, throw light upon the details of the art of warfare of that time. It may be doubtful, however, whether the soap-boiler of the present day could improve his own system by the author's.

He makes a curious remark on the nature of the ox and the horse, by a paradox, observing, that if oxen drink first, then there will be enough water for both oxen and horses; but, if the horses drink first, there will not be sufficient either for horses or oxen.

His instructions for writing in gold and silver letters are particularly interesting in the present day, when so much labour and ingenuity is expended in reviving, by fac-similes, ancient illuminated MSS.

It is observable that most of the countries, in which the materials quoted were produced, are in the East; for instance, Alexandria, Ethiopia, Persia, Cappadocia, Egypt, etc., shewing that the principal trade for these articles lay in that direction. Allusion is sometimes made to Greece and Spain, often to Italy, but rarely to Germany, France, or England. There are some passages which seem to indicate the work to have been compiled in Italy, or at least that these arts were originally developed there.

Among the more interesting facts, which the author relates, is one, which appears to me to fix a boundary to the question as to the origin of Painting in Oils, and to shew that it was, at all events, \emph{after} the period when this work was written by the author, that that art was discovered. The statement I allude to is this:--- In describing the mode of preserving pictures from injury by water he recommends the artist ``to lay over them a coating of oil of \emph{Cicini}.''\footnote{Chapter 109.} It is clear from this that pictures were not at that time painted with colours ground up with oil, for, if they had been so, they would not need a fresh coating to preserve them from water. Now it would be very extraordinary if a person, so conversant in the composition and the use of colours as this author seems to be, who must have had consequently much intercourse with painters, should be totally ignorant of the art of mixing the colours with oil, if it really was in general practice. I may adduce, in corroboration of this supposition, the picture of Lorenzo of Venice,\footnote{See Mrs. Merrifield's ``Cennini on Painting.''} painted in 1369, and described by Zanetti, which shows clearly that the practice, recommended by the author of Mappæ Clavicula, was still in use two hundred years after his own time; therefore, when it is stated that Theophilus mentions painting in oil as known before the eleventh century, the two statements seem to be irreconcilable.

I have, however, a MS. of the Latin Gospels of the tenth century, in which the figures of the Evangelists are deemed, by a lady skilled in painting, to have been done in oil. They are evidently of the Greek style of painting, as may be seen by comparison with other paintings in my Greek MSS. Should this conjecture be well-founded, it would be plain that Oil-painting must have been known \emph{before} the \emph{eleventh} century. Perhaps I may hazard a conjecture, that Theophilus (a Greek name) was a Greek artist, who came into Italy, and imparted the knowledge of his art to the monk Ruggiero, and that the author of the Clavicula, being an Englishman, and far distant from Italy, had not yet acquired that secret of mixing the oil and colours, which Theophilus may have known.

To return to the contents of the Clavicula.

Another object of interest is the explanation of the Symbols of Weights and Measures, which partly resemble those now in use, and approximate also to those given by Rabanus, Abbot of Fulda, in some cases, but not in all.

Near the end of the MS. he gives two or three cases of natural magic, one of which is the now well-known experiment of water quickly whirled round in a vessel, and yet not falling out.

One of the most valuable entries in the book is one connected with philology, namely, the alphabet of Runes, if they are Runes, but which I am inclined to think arc Oscan, or very early Greek, and derived originally from the Persepolitan, or Babylonian character. Those who will compare the earliest known Greek inscriptions, such as the Sigæan, will see at onee that they are made in the same cuneiform manner as the Babylonian and Persepolitan letters, and appear to have very nearly the same form of strokes or lines, with the advantage of a more simple combination of these strokes, so as to make one character or letter. The Chinese character retains its Babylonian form to this day; and the Runes of the North I consider to be equally derived from the Babylonian, although arbitrarily altered from the Southern form, perhaps according to the will and taste of the leader who led his bands northward after the dispersion of Babel, or perhaps at a later time by some Priest or Magus of the tribe, or they may be the same characters altered in the lapse of time, like some of our own from the ancient Saxon. Some of the letters in this volume correspond with the characters of the ``Lettras desconocidas,'' or unknown letters of Spain, as may be seen by referring to Laborde's Spain, Don Gabriel's Sallust, Velasquez, and the Oscan monuments. Those which correspond with Laborde are, ---

\begin{figure}[H]
\centering
\includegraphics[width=0.85\textwidth,keepaspectratio]{003-trans.png}
\caption*{h\hspace{17mm} i\hspace{17mm} l\hspace{17mm} s\hspace{17mm} t\hspace{17mm} q}
\end{figure}

\paragraph{}
The alphabetical names of ag, berch, cen, derhu, I do not remember to have met with before.

I have thus, my dear Sir, at your request, undertaken the above description of this volume, but I have felt myself totally incompetent to enter upon the task in such a manner as it deserves. To do it justice I consider could only be performed by some one, who, like yourself, had studied those arts which the book describes. I must request you therefore to pardon all faults, both of criticism, of judgment, and of simple description, and beg to remain,

My dear Sir, very truly yours, 

\bigskip

\textbf{Thos. Phillipps.}
\clearpage
\section{Incipit Libellus Dictus Mappæ Clavicula.}
\begin{quotation}
Sensim per partes discuntur quælibet artes.

Artis pictorum prior est factura colorum;

Post, ad mixturas convertat mens tua curas;

Tunc opus exerce, sed ad unguem cuncta coerce,

Ut sit ad ornatum quod pinxeris, et quasi natum.

Postea multorum documentis ingeniorum

Ars opus augebit, sicut liber iste docebit.
\end{quotation}
\subsection{De Vermiculo.}
\paragraph{}
Si vis facere Vermiculum, accipe ampullam vitream et lini deforis de luto, et sic accipe unum pondus vivi argenti, et duo pondera sulfuris albi aut crocei coloris, et mitte ipsam ampullam super 3. aut 4. petras, et adhibe ignem in circuitu ampulle ex carbonibus, ignem tamen lentissimum, et sic cooperies ampullam ex parvissima tegula: et, quando videris fumum exire ex ore ampullæ blavum, cooperi: et, quando exierit fumus crocei coloris, iterum cooperi: et, quando videris exire fumum rubeum quasi vermiculum, sic tolle ignem, et habes vermiculum optimum in ampulla.

\subsection{De Lazorio.}
\paragraph{}
Si vis facere Lazorium optimum, accipe ollam novam que nunquam fuit in opus, et mitte in eam laminas purissimi argenti, quantas vis, et sic cooperi ollam et sigilla; et mitte ipsam ollam in vindemia que est projecta de torculari, et illic bene cooperi de ipsa vindemia, et serva bene usque ad 15. dies; et sic aperies ipsam ollam, et illum florem qui est in circuitu laminarum argenti excuties in nitidissimo vase. Quod si amplius volueris habere, iterum fac quod supra scriptum est.

\subsection{Item.}
\paragraph{}
Si aliud Lazorium volueris facere, accipe ampullam purissimi cupri, et mitte in eam calcem usque ad medium, et sic imple illam fortissimo aceto; et ita cooperi et sigilla; et tunc mitte ipsam ampullam in terra, aut in alio aliquo calido loco, et ita dimitte usque ad unum mensem; et postea aperies ampullam. Istud Lazorium non est tam bonum sicut aliud, tamen valet ad lignum et maceriam.

\subsection{Item.}
\paragraph{}
Tertium Lazorium si vis facere, accipe flores blavos et tere, et exprime in mundissimo vase; et fac prius campum in ligno et in pargameno de albo plumbo; et mitte desuper, quando fuerit siccum, ipsum colorem, et tantum ita fac, usque quo videas ipsum colorem esse similem Lazorii.

\subsection{De Viridi.}
\paragraph{}
Si vis facere Viride Grecum, accipe ollam novam, et mitte in eam laminas purissimi cupri, et sic imple ipsam ollam fortissimo aceto, et ita cooperi et sigilla; et mitte ipsam ollam in aliquo calido loco, aut in terra, et ita dimitte usque ad sex menses, et tunc aperies ipsam ollam, et que in ea inveneris mitte super tabulam ligneam, et mitte ad solem siccare.

\subsection{Item.}
\paragraph{}
Si vis facere Viride Rotomagense, accipe laminas purissimi cupri, et lini ipsas laminas in circuitu de optimo sapone; et mitte ipsas laminas in novam ollam, et sic imple ipsam fortissimo aceto, et ita cooperi et sigilla, et (mitte) in aliquo calido loco usque ad 15. dies; et ita aperies ollam, et executies laminas super tabulam ligneam, et mittas ad solem siccare.

\subsection{De Minio.}
\paragraph{}
Si vis facere Minium, rubeum vel album, accipe ollam novam, et mitte in eam tabulas plumbeas, et imple ipsam ollam fortissimo acceto, et ita cooperies, et sigillabis; et mittes ipsam ollam in calido loco, et ita dimitte usque ad unum mensem; et postea accipe ollam, et discooperies, et quod fuerit in circuitu tabularum plumbearum excuties in alio vase fictili, et sic pones ad ignem, et semper movebis ipsum colorem, et quando videbis ipsum colorem effectum album, sicuti nix, tolles de illo quantum tibi placuerit, et ipse color vocatur Cerussa. Reliquum vero dimittes ad ignem et semper movebis, usque quo sit factus rubeus, sicuti aliud minium; et ita tolles de igne, et dimittes in ipso vase refrigerare.

\subsection{De diversis Coloribus.}
\paragraph{}
Colores in pargameno spissi, et clari, hii sunt: Azorium, Vermiculum, Sanguis draconis, Carum, Minium, Folium, Auripigmentum, Viride Grecum, Gravetum Indicum, Brunum, Crocus, Minium rubeum vel album, Nigrum optimum ex carbone vitis. Hii omnes colores destemperantur a glarea.

\subsection{De Mixtionibus.}
\paragraph{}
Quod si volueris scire naturas et mixtiones istorum colorum, et qui sint sibi contrarii, diligenter aurem appone.

Azorium misce cum albo plumbo, incide de indico, matiza de albo plumbo.

Vermiculum purum incides de bruno, aut de sanguine draconis, matizabis de auripigmento. Vermiculum misces cum albo plumbo, et facies colorem qui vocatur Rosa; incides de vermiculo, matizabis de albo plumbo. Item, facies colorem de sanguine draconis, et de auripigmento; incides de bruno, matizabis de auripigmento.

Carum minium incides de bruno, matizabis de rubeo minio. Item, facies Rosam de caro minio et albo plumbo; incides de caro minio, matizabis de albo plumbo.

Folium incides de bruno, matizabis de albo plumbo. Item, misces folium cum albo plumbo; incides de folio, matizabis de albo plumbo.

Auripigmentum incides de vermiculo, et ipsi matizatura non est, quia stercorat omnes alios colores.

Tamen si vis facere Gladum Viride, misces auripigmentum cum nigro; incides de nigro, matizabis de auripigmento.

Si vis facere similem, accipe azorium, misces cum albo plumbo; incides de azorio, matizabis de albo plumbo; et quando fuerit siccum cooperies de claro croco.

\subsection{Temperatura.}
\paragraph{}
Viride Grecum distemperabis cum aceto; incides de nigro, matizabis de albo, quod fit de cornu cervi. Item, misces viride cum albo plumbo; incides de viride, matizabis de albo plumbo. Gravetam incides de viride, matizabis de albo plumbo. Crocum incides de vermiculo, matizabis de albo plumbo. Indicum incides de nigro, matizabis de azorio. Item, misces Indicum cum albo, incides de azorio, matizabis de albo plumbo. Brunum incides de Nigro, matizabis de rubeo minio. Item, facies de bruno et albo plumbo Rosam; incides de bruno, matizabis de albo plumbo. Item, misces Crocum cum albo plumbo, incides de croco, matizabis de albo plumbo; minium rubeum incides de bruno, matizabis de albo plumbo. Item, misces Minium cum bruno; incides de nigro, matizabis de rubeo plumbo. Item, facias Carnaturam de rubeo plumbo et albo; incides de vermiculo, matizabis de albo plumbo.

\subsection{Qui contrarii sibi sint Colores.}
\paragraph{}
Modo si vis scire qui colores sibi sint contrarii, hoc est. Auripigmentum non concordat cum folio nec cum viridi, nec rubeo plumbo, nec cum albo. Viride non concordat cum folio.

Si vis facere campos, fac pulcram rosam de vermiculo et albo. Item, fac campum de folio distemperato cum calce. Item, fac campum de viridi distemperato cum acceto. Item, fac campum de ipso viridi, et quando fuerit siccum cooperies de caule.

Si vis scribere de auro, accipe pulverem auri, et distempera cum glute ipsius pargameni in quo debes scribere, et ad ignem de ipso auro cum glute scribe, et quando littera sicca fuerit, bruni de planissima petra, aut de dente apri. Item, si inde volueris vestimentum, aut picturam aliam facere, sicut superius dixi, aurum mittes in pargameno, incides de incausto, aut de indico, et matizabis de auripigmento.

\section{Incipit Prologus sequentis Operis.}
\paragraph{}
Multis et mirabilibus in hec meis conscriptis libris cure nobis fuit exponere commentarium, non ut tangentes sacros libros, et multum laborantes nichilque efficientes videamur, sed distinguentes istam heresim fatali munere concessam, omnem picturam et laborem que in ipsis actibus sint inveniamus ista volentibus perspicere. Appellantes quidem hujus compositionis cognominationem ``Mappæ Claviculam,'' ut omnis qui attigerit, multaque probaverit, existimet clavis modum esse inhibitum. Sicuti enim clausas domus sine clavi impossibile est facile patere iis qui in domibus sint, ita et sine isto commentario omnis scriptura que in sacris scripturis scripta est, clausum et tenebrosum sensum efficiet ejus qui legerit. Conjuro autem per magnum Deum, qui invenerit, nulli tradere nisi filio, cum primum de moribus ejus judicaverit, utrum possit pium et justum sensum habere, et ista conservare. Multa, vero, alia de virtutibus eorum, que scribuntur, habentes, dicere digna, incipiemus ab ipsis capitulis.

\section*{Explicit Prologus.}
\clearpage
\subsection*{Incipiunt Capitula.}
\begin{center}
    \footnotesize
    \begin{longtable}{|l|l|}
    \hline
        Aurum plurimum facere.                                                          & 1 \\ \hline
         Item, aurum facere.                                                             & 2 \\ \hline
         Item, aurum facere.                                                             & 3 \\ \hline
         Item, id ipsum.                                                                 & 4 \\ \hline
         Auri plurimi confectio.                                                         & 5 \\ \hline
         Auri confectio.                                                                 & 6 \\ \hline
         Secunda confectio.                                                              & 7 \\ \hline
         Auri confectio.                                                                 & 8 \\ \hline
         Item, auri confectio.                                                           & 9 \\ \hline
         Item, confectio.                                                                & 10 \\ \hline
         Item, auri plurimi.                                                             & 11 \\ \hline
         Item, aurum facere.                                                             & 12 \\ \hline
         Auri infectio.                                                                  & 13 \\ \hline
         Ex ere cornario quam oportet abscondere.                                        & 14 \\ \hline
         Auri confectio que non fallit.                                                  & 15 \\ \hline
         Auri alia confectio.                                                            & 16 \\ \hline
         Item, id ipsum.                                                                 & 17 \\ \hline
         Aurum viride facere conflatione.                                                & 18 \\ \hline
         Aurum probatum facere.                                                          & 19 \\ \hline
         Aurum gravius facere.                                                           & 20 \\ \hline
         Auri coctio.                                                                    & 21 \\ \hline
         Auri confectio.                                                                 & 22 \\ \hline
         Aurum gravius facere.                                                           & 23 \\ \hline
         Auri concilio et reformatio.                                                    & 24 \\ \hline
         Item, auri contio.                                                              & 25 \\ \hline
         Auri operacio.                                                                  & 26 \\ \hline
         Auri duplicatio.                                                                & 27 \\ \hline
         Aliter.                                                                         & 28 \\ \hline
         Aliter.                                                                         & 29 \\ \hline
         Aliter.                                                                         & 30 \\ \hline
         Aliter.                                                                         & 31 \\ \hline
         Aliter.                                                                         & 32 \\ \hline
         Aurum durum fusile facere quod melius ab igne exeat.                            & 33 \\ \hline
         Aurum solvere.                                                                  & 34 \\ \hline
         Aurum liquidum facere.                                                          & 35 \\ \hline
         Aurum mollire, ut in eo sigillum fingas.                                        & 36 \\ \hline
         Auri solutio.                                                                   & 37 \\ \hline
         Aurum ut solvi possit sine igne.                                                & 38 \\ \hline
         Aureas litteras scribere.                                                       & 39 \\ \hline
         Aliter.                                                                         & 40 \\ \hline
         Aliter.                                                                         & 41 \\ \hline
         Aliter.                                                                         & 42 \\ \hline
         Auri inscriptio secundum primam.                                                & 43 \\ \hline
         Alia.                                                                           & 44 \\ \hline
         Auri confectio.                                                                 & 45 \\ \hline
         Auri alia scriptio.                                                             & 46 \\ \hline
         Auri alia scriptio sine auro.                                                   & 47 \\ \hline
         Alia.                                                                           & 48 \\ \hline
         Alia.                                                                           & 49 \\ \hline
         Aurei coloris scriptura in cartis, in marmore, in vitro, ut videantur de auro.  & 50 \\ \hline
         Deauratio vitrorum in calamo et in ere.                                         & 51 \\ \hline
         Aureum colorem habere que voles.                                                & 52 \\ \hline
         Auri solutio ad picturam.                                                       & 53 \\ \hline
         Aurum viride facere.                                                            & 54 \\ \hline
         Deauratio omnium, si velis deaurare sive argenteum sive aureum vas.             & 55 \\ \hline
         Nigrum compingere, ut putes impisatum esse.                                     & 56 \\ \hline
         Deauratio stagnearum laminarum.                                                 & 57 \\ \hline
         Deauratio facilis.                                                              & 58 \\ \hline
         Item.                                                                           & 59 \\ \hline
         Ferrum auro conjungere.                                                         & 60 \\ \hline
         Idem, quasi decus.                                                              & 61 \\ \hline
         Argenti confectio.                                                              & 62 \\ \hline
         Eramentum candidum facere.                                                      & 63 \\ \hline
         Album eramentum facere ad auri infectionem.                                     & 64 \\ \hline
         Argenti confectio.                                                              & 65 \\ \hline
         Plumbum similem argento facere.                                                 & 66 \\ \hline
         Crisocolle confectio.                                                           & 67 \\ \hline
         Eris commutatio.                                                                & 68 \\ \hline
         Argenteis litteris scribere.                                                    & 69 \\ \hline
         Ex ere argentum, vel elidrium, vel aurum, facere.                               & 70 \\ \hline
         Quomodo oportet argentum nigrum candidum facere.                                & 71 \\ \hline
         Argenti mixtura.                                                                & 72 \\ \hline
         Argenti vasa tergere sine abusia.                                               & 73 \\ \hline
         Argentum aureo colore apparere.                                                 & 74 \\ \hline
         Candidi confectio.                                                              & 75 \\ \hline
         Liquidi argenti confectio per quam quis aurum deargentet.                       & 76 \\ \hline
         Argenti colorem saphirinum.                                                     & 77 \\ \hline
         Ut argentum et es auri colorem excipiat.                                        & 78 \\ \hline
         Argenti inscriptio Italica.                                                     & 79 \\ \hline
         De argento vivo scribere.                                                       & 80 \\ \hline
         Eris usum argento similem facere.                                               & 81 \\ \hline
         Colorem viridem facere.                                                         & 82 \\ \hline
         Indicum colorem facere.                                                         & 83 \\ \hline
         Collam Grecam facere.                                                           & 84 \\ \hline
         Inauratio lapidis, vel ligni, vel vitri.                                        & 85 \\ \hline
         Ad colorandum aurum.                                                            & 86 \\ \hline
         De dispositione fabrice.                                                        & 87 \\ \hline
         De fabrica in aqua.                                                             & 88 \\ \hline
         De multa.                                                                       & 89 \\ \hline
         De licamonia.                                                                   & 90 \\ \hline
         De compositione cinnabarin.                                                     & 91 \\ \hline
         De compositione Iarin.                                                          & 92 \\ \hline
         De compositione psimitii.                                                       & 93 \\ \hline
         Compositio Lazurin.                                                             & 94 \\ \hline
         Ut pictura aqua deleri non possit.                                              & 95 \\ \hline
         Confectio pandii.                                                               & 96 \\ \hline
         Confectio ficarin.                                                              & 97 \\ \hline
         Deauratio in ligno vel panno.                                                   & 98 \\ \hline
         De lineleon.                                                                    & 99 \\ \hline
         De lineleon in exauratio.                                                       & 100 \\ \hline
         De inductione exaurationis petalarum.                                           & 101 \\ \hline
         Tinctio stagnææ petale.                                                         & 102 \\ \hline
         Confectio crisocolle.                                                           & 103 \\ \hline
         Aliud crisocollon.                                                              & 104 \\ \hline
         Item, ad ipsum.                                                                 & 105 \\ \hline
         Item, aliud.                                                                    & 106 \\ \hline
         Eramenti gluten.                                                                & 107 \\ \hline
         Gluten de ligno vel osse.                                                       & 108 \\ \hline
         De metallo auri ad coctionem.                                                   & 109 \\ \hline
         De metallo argenti.                                                             & 110 \\ \hline
         De lapide adamante.                                                             & 111 \\ \hline
         De conchilio tinctio porfirii.                                                  & 112 \\ \hline
         De porfiro\footnote{\emph{Sic}: in the chapter corresponding to this title this word is written porfirio.} citrino.                                                            & 113 \\ \hline
         De crisorantida.                                                                & 114 \\ \hline
         De auri sparsione.                                                              & 115 \\ \hline
         De oxiporfironto aporodinis.                                                    & 116 \\ \hline
         De porfiro citrino.                                                             & 117 \\ \hline
         De argissorantista.\footnote{\emph{Sic}: in the chapter this word is written argirosantista.}                                                         & 118 \\ \hline
         De argenti sparsione.                                                           & 119 \\ \hline
         De smiria petra.                                                                & 120 \\ \hline
         De terra limia.                                                                 & 121 \\ \hline
         De lapide focario.                                                              & 122 \\ \hline
         De lapide fisso.                                                                & 123 \\ \hline
         Quomodo fiat cera marmor ex gagate.                                             & 124 \\ \hline
         De lapide tracho.                                                               & 125 \\ \hline
         Compositio electri.                                                             & 126 \\ \hline
         Gluten auri ad fistulas.                                                        & 127 \\ \hline
         Compositio litargiri ex plumbo.                                                 & 128 \\ \hline
         Alia compositio litargiri ex argento.                                           & 129 \\ \hline
         Inauratio musii operis.                                                         & 130 \\ \hline
         De tabulis smirutatis.                                                          & 131 \\ \hline
         De coloratione musii.                                                           & 132 \\ \hline
         Quian ita fiet.                                                                 & 133 \\ \hline
         Cathmie compositio.                                                             & 134 \\ \hline
         Anfinus sic fit.                                                                & 135 \\ \hline
         Pandii compositio.                                                              & 136 \\ \hline
         Alia.                                                                           & 137 \\ \hline
         Aliter.                                                                         & 138 \\ \hline
         Tinctio vitri prassina.                                                         & 139 \\ \hline
         Alia.                                                                           & 140 \\ \hline
         Tinctio sanguinea.                                                              & 141 \\ \hline
         Tinctio alithina absque igne.                                                   & 142 \\ \hline
         Minus tincta melini colons.                                                     & 143 \\ \hline
         Antimis de damia.                                                               & 144 \\ \hline
         De lapide olimpio.                                                              & 145 \\ \hline
         De lapide flavite.                                                              & 146 \\ \hline
         De lapide rubeo.                                                                & 147 \\ \hline
         Compositio lulacis.                                                             & 148 \\ \hline
         Compositio lazurin.                                                             & 149 \\ \hline
         Lazurin aerium.                                                                 & 150 \\ \hline
         Lazurin carnei coloris.                                                         & 151 \\ \hline
         Lazurin melinizonta.                                                            & 152 \\ \hline
         Aliud lazurin.                                                                  & 153 \\ \hline
         Compositio vermiculi.                                                           & 154 \\ \hline
         Pandius.                                                                        & 155 \\ \hline
         Item, pandius.                                                                  & 156 \\ \hline
         Pandius, ocrei coloris.                                                         & 157 \\ \hline
         Compositio nigelli.                                                             & ~ \\ \hline
         Item, nigellum ad almenbuz.                                                     & 158 \\ \hline
         Item, si vis aurum ponere in pellem.                                            & 159 \\ \hline
         Si vis colorare almenbuz.                                                       & 160 \\ \hline
         Si vis nectere eramen, aut auricalcum.                                          & 161 \\ \hline
         Connexio auricalci.                                                             & ~ \\ \hline
         De stagno conjunctio.                                                           & 162 \\ \hline
         Deauratio facilis.                                                              & 163 \\ \hline
         Ad gluten stanni.                                                               & 164 \\ \hline
         Inaurato vase nigrum impingere, ut putes inpistatum esse.                       & 165 \\ \hline
         Inductio exaurationis petalarum.                                                & 166 \\ \hline
         Tinctio stagnææ petale.                                                         & 167 \\ \hline
         Aurum probatum faccre.                                                          & ~ \\ \hline
         Ad solidaturam de argento.                                                      & 168 \\ \hline
         Ad solidaturam argenti non boni.                                                & 169 \\ \hline
         De lapide orcho.                                                                & 170 \\ \hline
         De atriathe.                                                                    & 171 \\ \hline
         De lapide fumice.                                                               & 172 \\ \hline
         Compositio auripigmenti.                                                        & 173 \\ \hline
         Gluten auri ad fistulas.                                                        & 174 \\ \hline
         Crisopargia de petalis.                                                         & 175 \\ \hline
         Quomodo fiat sulfur coctum.                                                     & 176 \\ \hline
         Confectio ficarin.                                                              & 177 \\ \hline
         De metallo vitri et coctione.                                                   & 178 \\ \hline
         De metallo plumbi.                                                              & 179 \\ \hline
         Alia coctio plumbi ex ipso metallo.                                             & 180 \\ \hline
         Alia compositio vitri.                                                          & 181 \\ \hline
         Qualiter pelles tinguantur.                                                     & 182 \\ \hline
         Pellis rubea tinctio.                                                           & 183 \\ \hline
         Prassine pellis tinctio.                                                        & 184 \\ \hline
         Item.                                                                           & 185 \\ \hline
         Melina pellium tinctio.                                                         & 186 \\ \hline
         De porfirio melino.                                                             & 187 \\ \hline
         De prima tinctione pandii.                                                      & 188 \\ \hline
         De secunda tinctione pandii.                                                    & 189 \\ \hline
         Item, de pandio.                                                                & 190 \\ \hline
         Item.                                                                           & 191 \\ \hline
         Tinctio prassina ossium, cornuum, et lignorum.                                  & 192 \\ \hline
         De veneti tinctione eorundem.                                                   & 193 \\ \hline
         De melina tinctione eorundem.                                                   & 194 \\ \hline
         De colore cinnabarin.                                                           & 195 \\ \hline
         De cebellino quomodo fiat.                                                      & 196 \\ \hline
         De inauratione ferri.                                                           & 197 \\ \hline
         De petalo aureo.                                                                & 198 \\ \hline
         Lucida quomodo fiant super colores.                                             & 199 \\ \hline
         De crisografia.                                                                 & 200 \\ \hline
         De inauratione pellis.                                                          & 201 \\ \hline
         Confectio maltæ.                                                                & 202 \\ \hline
         De lacca quomodo laboratur ad pingendum ligna, seu parietem.                    & 203 \\ \hline
         De calce et arena.                                                              & 204 \\ \hline
         De latericiis parietibus.                                                       & 205 \\ \hline
         Confectio saphiri.                                                              & 206 \\ \hline
         Confectio vitri rubei.                                                          & 207 \\ \hline
         Alio modo.                                                                      & 208 \\ \hline
         Gluten argenti et auri.                                                         & ~ \\ \hline
         Aliud gluten stagni.                                                            & 209 \\ \hline
    \end{longtable}
\end{center}
\subsection*{Expliciunt Capitula.}
\clearpage
\section{\emph{Incipit liber dictus} Mappe Clavicula.}
\subsection{Aurum plurimum facere.}
\paragraph{}
Sumes argentum vivum \roundz. 8. limature auri \roundz. 4. et ex argenti boni limatura \roundz. 5. eris cipri limature \roundz. 5. auricalci limature \roundz. 5. aluminis scissi et floris eris, quod Greci ``Calcantum'' vocant, \roundz. 12. auripigmenti coloris aurei \roundz. 6. elidrii \roundz. 12. et tunc miscebis omnes limaturas cum argento vivo, et facies in modum ceroti; et mittes elidrium et auripigmentum simul; deinde eris florem et alumen adiciens; omnia in patina super carbones pones, et leviter coques, aspergens desuper manu crocum acceto infusum, et nitri modicum, et croci quidem \roundz. 4. minutatim asperges donec resolvatur, et patieris ut combibat; cum, autem, coagulatum fuerit, tolle, et habebis aurum cum cremento. Adicies autem superioribus speciebus etiam terre lunaris modicum, que Grece dicitur ``Affroselinum.''

\subsection{Item, Aurum facere.}
\paragraph{}
Argenti \roundz. cipri \roundz. dimidiam, et auri \roundz. 1. confla. Item, accipies arenam sub aream teris, et refrigeras donec siccetur, et commisces rursum sale, et assabis in fornace die et nocte; postea eximis et lavas donec sal effluat; et rursus siccas et macerabis aceto, et dimittes modicum donec combibat et siccetur; deinde rursus mittes in fornacem speciem non lotam, et hoc facies semel atque iterum; macerabis acceto quotiens in fornacem mittis. Mittere autem in fornacem debes quater vel quinquies, donec fiat quasi subcoctum, et cum tuleris, sumes tracturam argenti que, Grece, ``Elquison'' dicitur, equale pondus mensure superiori, et mixta omnia teris; deinde conflabis separatim, et paulatim ex ambabus speciebus eum quas confecisti sparges donec consumatur; deinde refrigeras, et invenies plumbum durum effectum. Hoc cum cepsonio, quod est cinis maceratus, confla de folle. Ut demonstret se secundum claviculam. Psomion est cinis aqua maceratus quem substernis in fornace ad digiti crassitudinem.

\subsection{Item.}
\paragraph{}
Accipies ad experimentum donec primitus discas non multum cum semel facias. Accipies cupri eris posios rufi \roundz. 1. argenti \roundz. 1. saponis sene boni, et confla paleis donec productum non crepet, et tunc confla cum eo auri \roundz. 1. et nitri tantundem: postea convertis vatillos duos faciem ad faciem (id est, duas cavatas testas) et ponis in medio conflaturam latificatam (id est, late productam) que est preparata, et antisma commisces, quod est secundum barbillum eris jam curvati in barbillo, argenti \roundz. 4. in barbillo invenimus auri magis equam portionem, sinopidis Pontici, partem unam, salis comunis partes 2. Que omnia simul terens, laminas substernis, et superaspergis, et fictili creta oblinis, ne respiret, et subice ignem donec estimes bene habere: aufer, habebis aurum optimum.

\subsection{Item.}
\paragraph{}
Argenti partes 4. misii cipri partes 4. elidrii contusi et cribellati partes 7. sandarace partes 4. miscebis; et conflabis argentum, et asperges species superiores, et vehementi igne confla, commovens omnia pariter, donec auri colorem videas; et, eximens, extingue in aqua frigida, in cratera habentem hujus confectionem commixtionis. Misii cipri, et sandarace, et elidrii partes equales; et facies pinguedinem mollem, confla argentum, et cum adhuc calidum habetur, infunde in eandem pinguedinem.

\subsection[5. Auri plurimi confectio.]{5. Auri plurimi confectio.\footnote{To this a later hand has written ``\emph{Nota bene}.''}}
\paragraph{}
Accipies ciprum quod calide productum sit, \emph{id est}, ejus limaturam, teris in aqua cum auripigmenti crudi partibus 2. ut fiat glutinis pinguedo, et assa in cacabulo horis 6. fiet nigrum; hoc tollis, et abluis, deinde mittis salem equalem portionem, et teris pariter: deinde assa in cacabulo, et vide quid fiat. Si enim album fuerit, misce argentum; si flavum, misce aurum, equali portione, et fiet mirabile.

\subsection{Auri confectio.}
\paragraph{}
Accipies fellis hircini partes 2. fellis taurini 1. et elidrii triplum ad superiores species; et teris cum acceto diebus 10. Deinde sumis crocum Licium vel Arabicum, et teris in mortario Thebaico in sole diebus canicularibus; adicies et accetum acerrimum, et tamdiu teris donec se crocum dimittat, et consumatur. Mittis autem aceti sextarium, et melius si plus, et patere siccari. Deinde accipies batracium metallicum, (quod alii ``ciprum'' vocant,) tere minutissime, et admisce crocum; deinde tolles crisantimum, (\emph{id est}, aurum) quo pictores utuntur, (quod etiam pusam vocant), et tere similiter, et habebis in usum. Hoc autem facies ita. Sumes argentum quantum voles, et conflabis, et adicies salem tenuem tritum, et commovebis diutius donec solvatur argentum, postea pateris coagulari, et, cum adhuc fervet, diffundis in aquam marinam. Deinde iterum conflabis, et de prima compositione mittis, in libram unam, argenti \roundz. 2. et commovebis fortiter, et rursum dimittes coagulari: deinde iterum mittis et conflas et adicias de secunda compositione $\svgaaa$ 1. Similiter commovebis plurimum, et rursum defundis. Item, tercio conflas, et adicias de tercia confectione sextarium, et habebis. Commove autem ferro novo quotiens commoveris, vel videris; et, ubi tibi visum fuerit accesisse in unum, admisces rursus argentum donec tibi videatur bene habere.

Confectiones iste sunt. Prima hæc.

Accipies fellis hircini partes 2. fellis taurini partem 1. elidrii triplum ad superiores species, id est, partes novem, et teris cum acceto diebus 10.

\subsection{Secunda confectio.}
\paragraph{}
Crocum Licium vel Arabicum teris in mortario Thebaico cum acceto acerimo in sole diebus canicularibus, donec se crocum dimittat et consumatur. Mittis autem acceti sextarium 1. et melius si amplius, patere siccari; deinde accipe batracium, (quod est ciprum,) tere minutissime, et admisce crocum.

Confectio tertia. Accipe crisantimum quo pictores utuntur, (quod etiam pusam vocant,) et tere similiter. Has autem confectiones habebis impromptu, et secundum ea que supradicta sunt, de prima confectione mittis in libram unam argenti \roundz. 1. in $\svgaaa$ vero scripula 2. De secunda autem confectione vel compositione mittis in libram unam argenti \roundz. 1. in $\svgaaa$ vero argenti scripulum 1. Item, de tercia confectione in libram argenti mittis \roundz. 1. in $\svgaaa$ vero argenti siliquas 3.

\subsection{Auri confectio.}
\paragraph{}
Ex argento puro producis laminas plures, et substernis eis medicamen quod declarabitur, et super asperges, et conflabis donec in unum omnia conducantur. Est autem hoc medicamen, quod ``offa'' appellatur: accipies auri scripula 4. glutinis Macedonicii \roundz. 1. sulphuris vivi \roundz. 1. nitri \roundz. 2. minii Hispanii \roundz. 1. fellis vulpis totum, elidrii $\svgaaa$ 1. croci Licii $\svgaaa$ 1. et facis potionem ferream, in qua hec universa mittis, et ex hoc medicamine ut superius, et laminas substernis, et desuper aspergis. Mittis autem in libram 1. argenti medicaminis $\svgaaa$ 1. et conflas, et erit aurum.

\subsection{Item.}
\paragraph{}
Sumes aurum quantum volueris, et duplum miseos peregrini, et scobem prinapiam eris boni quantum et miseos, aut eris cipri conflati; permiscebis utraque, et facies de auro tubulum, atque in eodem deponis tria medicamina; et sic conflabis, refrigens ut oportet, et eximis de fornace et abluis, et invenies auri plurimum pondus; quod cum ignem tetigerit, melius fiet. Cathmiam, aut trachiam, (est aut flava et laminosa, id est, lata,) vel calactiten per misce, donec tibi aurum exactum videatur.

\subsection{Item, auri confectio.}
\paragraph{}
Piritis lapidis (\emph{id est}, petre focarie) tolle partes 2. plumbi boni partem 1. confla pariter donec ut aqua fiet: posthac adice plumbum in fornace donec bene misceantur: postea tolle extra, et teris ex his partes 3. et calcitis boni partem 1. simul teris et assas donec flava fiant, et conflabis eramen quod ante purgaveris, et adicies ex medicamine ad oculum (id est, ad existimationem). Fiet aurum.

\subsection{Item, Auri plurimum.}
\paragraph{}
Sumes auri primi quartas 3. et eris cipri quartam 1. simul conflabis, et limabis lima aurificis, et adicies argenti vivi \roundz. 2. simul teris, et adicies accetum modicum, et modicum salis, donec combibat argentum vivum limaturam, et fiet malagma; et dimittis siccari diebus 7. in ampulla vitrea. Postea sumes sulfuris vivi preparati siliquas 4. et sandarace preparate, ex pusca salsa, et bubali lutei, siliquas 2. auripigmenti, quod de Scithico atramento fit, et fellis vulturis, siliquam 1. Pariter conteris, substernis huic malagmati; et latificabit malagma quod residuum est medicamentum super malagma; et obturabis diligenter os ampulle, et gipso oblinies, et assabis in superiore dispositione fornacis diebus et noctibus 3. quarta autem die transfer ampullam in inferiorem dispositionem fornacis, ut fiat quasi flavum, et tolle, et depone, et sumes argenti ciatos 3. et procurati auri 4. ta unam et conflabis, atque invenies quomodo se habeat, sanctum laudabile que secretum.

\subsection{Item, aurum facere.}
\paragraph{}
Sumes auri ciatos .5. et eraminis procurati ciatos 2. et conflabis pariter, et limabis tenuiter; et mitte argenti vivi, quod de minio factum sit, quartas 12. et conteris limaturam, et adicies aceti acerrimi et salis modicum, donec argentum vivum combibat limaturam, et fiet malagma; et sinito coqui diebus 7. Est autem medicamen hoc. Sulfuris siliqua\footnote{Siliquam, MS.} 1., sandarace siliqua 1., auripigmenti, quod de Scithico atramento fit, et fellis vulturini, siliqua 1. Hec omnia simul teris, et substernis malagma in ampulla; et oblinies gipso orificium ampullæ, et assabis in superiore dispositione fornacis, donec fiat subflavum; et tolles argenti, quod dicitur signati, quartas 4. et auri, quod est in ampulla, quartas 4. Simul conflabis, et invenies.

\subsection{Auri infectio ex ere cornario quam oportet abscondere.}
\paragraph{}
Cipri partem 1. fellis taurini partem 1. miseos assati partem 1. teris, calefacis, et invenies.

\subsection{Auri infectio que non fallit.}
\paragraph{}
Auripigmenti seissilis \roundz. 1. sandarace rufe pure \roundz. 4. corporis magnesie \roundz. 4. atramenti Scithici \roundz. 1. Greci nitri, ad similitudinem nitri accidentis, \roundz. 6. tollis; teris auripigmentum valde tenuiter, donec fiat quasi fuligo; et commisce omnia, et adice acetum Egiptium acerrimum, et fel taurinum; et una contere, et fac lutusum, et sicca in sole diebus tribus. Tere, et repone in ampulla; et assa, in qua nosti fornace, diebus 5. Postea tollis, et teris, adjecto gummino trito .\roundz. 5. adicis aquam et facis lutusum, et formabis collirium, et sumis auri primi partem 1. et collirii partem 1. Confla aurum, et mitte in collirium; et, cum factum fuerit aurum viride, et quod teri possit, infecti auri partem 1. et argenti 1. conflabis simul, et invenies aurum. Si velis id primum facere, infecti auri partes 4. et comunis partem 1. simul conflabis, et invenies aurum optimum et probatum. Absconde sanctum, et nulli tradendum secretum, neque alicui dederis propheta.

\subsection{Auri alia confectio.}
\paragraph{}
Eris partes 4. argenti partem 1. simul confla, et adice auripigmentum non ustum (\emph{hoc est}, crudum) partes 4. et cum valde calefeceris, sinis ut refrigeret, et mitte in patinam; obline argilla et assa, donec fiat cerosa: tolle et confla, et invenies argentum. Si autem multum assaveris, fit elidrium. Cui si partem 1. auri addideris, fit aurum optimum.

\subsection{Item, id ipsum.}
\paragraph{}
Sumis aurum ita factum, in laminas producis, ad unguis crassitudinem, et accipies sinopidis Egiptie partem 1. et salis partes 2. Simul misces, et substernis laminam; et, cum substernis, argilla obturabis, et assabis horas 3. Deinde tolles, et invenies aurum optimum, et sine aposia.

\subsection{Aurum viride facere, conflatione, sine conflatione.}
\paragraph{}
Aluminis liquidi partem 1., amomi Canopice, (qua aurifices utuntur,) partem 1., auri partes 2. hæc omnia confla; vide quid fiat.

\subsection{Aurum probatum facere.}
\paragraph{}
Armenii partes 2., zonitidi partem 1. Tere omnia, adice lutinis taurini quartam partem, cathmie equam. Confla, et erit gravius. Hoc idem facies et in ere.

\subsection{Auri coctio.}
\paragraph{}
Salis triti, et aluminis scissi, partes equales, misce ad $\div$. 1. scripula 3. misii ciprii conteris vatillum, pouum\footnote{Sic.} prepara; facis auri laminas tenues; componis unam inter unam; reples carbonibus, et incendis \~{t}.

\subsection{Auri confectio.}
\paragraph{}
Ferri eruginis $\svgaab$. 5. lapidis magnetis $\svgaab$. 5. aluminis extranei $\svgaab$. 5. mirre \roundz. 2. auri aliquid, teris cum vino; valde utile est. Sunt autem aliqui qui non credunt quanta sit utilitas ex humoribus, hi qui non per se demonstrationem faciunt. Oportet autem facientes divinis mirabilibus concedere omnia. Oportet facere sic per mixtionem confusa, et in fornace aurificis missa; follibus adhibitis ejus natura inveniatur.

\subsection{Aurum gravius facere.}
\paragraph{}
Operatio ex auro, ne vanus sit labor curaque, perficiendi, sed et cum lucro, commixtionem facere utilitas non modica erit. Cum aurum ignei coloris aspectum acceperit, accipies ferri Indici limaturas denariorum 2. teris optime, et mittis in aurum; delude battis laminam, ut non sit pingue. Utere autem misio et absconde in aliud; cumque visum fuerit sufficienter habere, calefacies modicum, ut focum tetigisse tantum videatur. Inde autem cum tuleris, in novam patellam mittis, et operies. In coctura vero laminæ utere misio. Exercens autem purgabis, et ut voles ita expendes, et quantumcunque fuerit, misce nitri partes 2. et plumbi 1. molipdinis partem 1. Permisces aqua pinguedinem; componis super folia, aut corticcm tenuem, et, ubi siccaveris, conflabis eam cum molipdino, et dimittis donec inurantur: remanet autem quod vis.

\subsection{Auri concilio et reformatio.}
\paragraph{}
Accipies auri $\svgaab$. 5. et facies fistulas, et mittes auricalci limaturam equali modo, et alumen sissum, et misium ciprium, et sal montanum; et confla, donec separentur omnia a se: et, cum extenderis medicamina inde excusseris, mitte fistulam in conflatorium, et nitrum Thebaicum, et ita conflas; et dispendens invenies plumbum effectum, quod, in ignem missum, et cesum, eundem colorem reddat.

\subsection{Item, auri conflatio.}
\paragraph{}
Auri partes 2. argenti 2. lamine ciprie partem 1. confla.

\subsection{Quomodo oportet auri probationem facere.}
\paragraph{}
Argenti limature partes 4. cathmie, sinopidis misii, et eris usti partes singulas; omnia simul teris, cum muro (vino) lavas; et, cum permixtio pura fuerit, facis offulam, et erit bolarum conflatum et inunctum. Cum hoc confla auri partes 4.

\subsection{Auri operatio.}
\paragraph{}
Auri partes 5. lamine ciprie partem 1. confla; argenti limature partes 4. commisces supradictis speciebus, simulque omnia conteris cum vino et lavabis; et, cum hec permixtio pura facta fuerit, fac offulam, et erit bolarii in moum.\footnote{? pro ``modum.''} Huic conflato adjungis auri partem 1. et ita simul conflabis.

\subsection{Auri duplicatio.}
\paragraph{}
Auri $\svgaab$ 4. misii $\svgaab$ 5. sinopidis $\svgaab$ 5. prepara, confla aurum, donec hilare fiat, admisce illa duo, misium et sinopidem, in conflatura, et tolle.

\subsection{Aliter.}
\paragraph{}
Auri partem 1. argenti similiter 1. eraminis 1. Fac laminam ad unguis crassitudinem: huic substernis, et supra adicis infectionem misii assi partem 1. Assas horis duabus; tolle, et invenies aurum duplicatum.

\subsection{Aliter.}
\paragraph{}
Auricalci primi limature partem 1. ut facile confletur, cathmie Samie minas 8. misii assi partes 8. ad e\footnote{? pro ``adde.''} minas 12. prepara: confla cum hac permixtione diligenter.

\subsection{Aliter.}
\paragraph{}
Opocarpasum misce 1. arborinum, quod est lacrima, quasi gummi, de arbore in qua arborinum nascitur. Aliqui autem volunt herbam vel arborem in Egipto.

\subsection{Aliter.}
\paragraph{}
Minium, et harenam montanam, auri limaturam, et alumen, cum aceto simul teris, et coquis in vase ereo, et commoves. Hujus scripture color per annos durat.

\subsection{Aliter.}
\paragraph{}
Nitri rufi $\svgaab$. 2. minii $\svgaab$. 3. permisce, et tere cum aceto, et adice modicum aluminis, et dimitte ut siccetur. Deinde tere, et repone: et sumis auri limaturam, ad dimidium obolum, et auripigmenti, quod aurei coloris fit, $\svgaab$. 1. Misces omnia et teris, et defundis gummi purum in aqua infusum, tolle, et signa que velis, sive epistolam, sive tabulas. Dimitte biduo, et coagulatur sigillum.

\subsection{Aliter.}
\paragraph{}
Depone aurum in succo juniperi, qui bero ``inbriome'' dicitur.

\subsection{Aurum durum fusile facere, quod melius ab igne exeat.}
\paragraph{}
Aurum obrizum limas lima tenui, et comminue diligenter; et in mortario ophitis, sive porfiretico aspero mittes, ubi teri bene possit: et adicies acetum acerrimum, et teris pariter, diluis quamdiu nigrum fuerit, et fundis. At ubi acetum colorem suum habuerit, tunc demum mittis aut sal granum, aut certe affronitrum, et sic solvitur ut de eo scribi possit. Tunc condis in dolio vitreo, et gummi modicum ut teneat, sicut in minario genere. Ita et argentum, et auricalcum, et ferrum, solvi possunt. Ut autem aurum, cum scripseris, possit lucere, coclea marina, vel aprino dente, literas perfrica.

\subsection{Aurum solvere.}
\paragraph{}
Melanei partem 1. auripigmenti partes 3. auri partem 1. tollis, sed aurum producis diligenter, quantum potes, forficabus, et adicies argenti vivi partem 1. aceti acerrimi partem 1. et omnia simul commisce: tere in mortario diligenter, et adicies gummi, et scribe in atramento, et dente sp(l)endifica.

\subsection{Aurum liquidum facere.}
\paragraph{}
Auripigmenti coloris aurei partem 2. elidrii partem 1. spume argenti, cujus color sit aureus, partem 1. Hec, cum triveris, diffunde in vase; postea accipies laminas aureas 24. ad quartam; quantum voles ex his, tere in mortario mundo medicinali, adjecto sale modico; et cum tibi apparuerit ut arena trita, adice aquam puram, et teris et ablues. Sed ita uti, frequenter adicies aquam, donec tibi aurum purum appareat; et tunc adicias ista medicamina: gummi tritum modicum, ita ut non sit glutinosum. Distillas destillatione croci, et omnia simul teris, ut sit atramenti pinguedo: et recipe aut in concam aut in vitreum vas; atque, cum uti voles, ungue prius cannam liquido alumine, deinde intingue in aurum, et scribe; et, cum siccaverit, dente frica diligenter.

\subsection{Aurum mollire, ut in eo sigillum fingas.}
\paragraph{}
Alumen scissile in aqua decoques, et sumis argentum vivum, et in mortario simul teris, aspergens ochre partem 1. et crocum cum glutine puro, et felle vitulino; et conteris, et utere.

\subsection{Auri solutio.}
\paragraph{}
Sumes stagnum, conflabis cum argento vivo, et dimittis ut refrigeret; teris in mortario diligenter cum alumine scissili, et lotio pueri investis. Sic fiet liquidum, et, cum fuerit atramenti scriptorii pinguedinem, scribe ex eo opus: cum siccaverint littere, separatim teris crocum Licium cum glutine puro; scribe ex eo que jam scripseras, et, cum siccaverit, dente fricas. Si vero infriguerit stagnum, reconflas, et ipsum argento vivo permisces.

\subsection{Aurum ut solvi possit sine igne.}
\paragraph{}
Sumis laminas aureas, et argenteas; teris in mortario durissimo cum sale, et nitro Greco, donec non compareant, et visum tibi fuerit diligenter tritum. Deinde adicis aquam aliam similiter, et abluis; et, ubi purum reliquum fuerit in mortario aurum, eris florem modicum, et fel taurinum conteris, et sic scribis. Recipe in vas vitreum, et, cum siccaverit, teris, ut hilariora fiant. Scribe autem calamo aut pincello pictorio. Si velis ut diffusum sit abundantius quod scribas, conteris cum his, que supra scripta sunt, separatim, auripigmenti scissilis partes 4. elidrii partem 1. cribrans et miscens. Conteris cum predictis tantum quantum tibi videatur esse equale ei auro quod in mortario est; et conteris, ut predictum est, et scribe. Cum siccaverit, delinis delitoria lana, vel nitro; pingis autem cum hoc et in vitro, et in marmore, et in imaginibus.

\subsection{Aureas litteras scribere.}
\paragraph{}
Sumis auream laminam productilem, forficabis ad minutum, et mittis in vas vitreum, et adicis argentum vivum, quod sufficiat, et dimittas, donec liquescat aurum, et transferes in morarium. Tere diligenter argentum vivum. Cum autem tibi visum fuerit valde tritum, adice misium, et ciprum, et similiter tere, donec non compareat argentum vivum. Tunc, misium mitte, donec tibi appareat cyprum, et dimitte ut siccet; et, adiciens gluten liquidum sufficienter, tere, et scribe pincello quod vis in coloribus, ut nosti.

\subsection{Aliter.}
\paragraph{}
Plumbum conflas frequenter, et intinguis in aquam frigidam; et tunc conflabis aurum, et restinguis in predicta aqua de plumbo, et fit fragile. Deinde teris diligenter aurum cum argento vivo. Ipsam autem fecem, cum quo scis, diligenter purgas, et misces gummi liquidum, et scribe; antea in alumine liquido calamum tingue. Sale et aceto purges alumen.

\subsection{Aliter.}
\paragraph{}
Sanguine draconis Indici intingue aurum, et pone in vitreo vase, et circumda aforis carbonibus; et statim solvitur, in tantumque erit liquidum, ut ex eo possis scribere.

\subsection{Aliter.}
\paragraph{}
Sumis stagnum, et digitis confricas; et, cum tibi nigrescere ceperint, confrica ex eis aurum, donec assumat eandem nigredinem; et tunc confla, postea refrigera, tere, et fac ut nosti.

\subsection{Auri inscriptio, secundum primam.}
\paragraph{}
Elidrii partem 1. resine frixe partem 1. ovorum numero 5. humorem, gummi partem 1. auripigmenti aurei coloris partem 1. fellis testudinis 1. limaz carcale partem 1. Sit autem sic eorum (\emph{id est}, tunsorum) omnium, pondus, ad \roundz. 20. Deinde adicias croci \roundz. 2. Facit autem hoc, non solum in cartis et in membranis, sed etiam in marmore et in vitro.

\subsection{Alia auri confectio.}
\paragraph{}
Sulfur vivum, corium mali Punici, ficorum interiora, aluminis scissilis parum, gummi liquidum misces; et scribe, adjecto croco modico.

\subsection{Item alia confectio.}
\paragraph{}
Ovorum trium vitellos, et unius\footnote{The words ``et unius'' are repeated in MS.} alborem, et gummi $\div$ 4. et croci $\svgaab$. 1. et cristalli limature $\svgaab$. 1. auripigmenti aurei coloris $\svgaab$. 7. Hec omnia diligenter teris; siccabis biduo subjecto, et tunc remittuntur croco. Sic scribis que velis.

\subsection{Auri alia scriptio.}
\paragraph{}
Elidrii partem 1. auripigmenti partem 1. fellis testudinis partem 1. aluminis scissilis 1. et de corio mali Punici, quod intus est aurei coloris 1. gummi 1. ova 5. Sit autem sic eorum omnium pondus $\svgaab$. 20. croci \roundz. 2.

\subsection{Auri alia scriptio, sine auro.}
\paragraph{}
Lac siccamine, vel ficus, et aluminis quartam partem ad lac, decoque in vase; in eo perunge, et sic inaura vas ipsum quod inaurabis; perunge ante mura quanta, aut lacte siccamine, aut ficus, et sic inaura.

\subsection{Alia.}
\paragraph{}
Nitrum rus(s)um et sal uná conflabis, et cum aqua inungues, et quod vis operis facias.

\subsection{Alia.}
\paragraph{}
Accipies glebam auri, mittis vas vitreum, et adice fel taurinum; spissa diligenter, et dimitte triduo. Deinde, cum didiceris quod dissolutum est, defunde fel sensim, et adice auro soluto aquam salsam, et transfers utrumque in ereum vas purum, et tepefac. Depone et dilue; et, cum siccaverit, misce liquidum gluten, et sic pinge. Sane tribus diebus non fuerit solutum, dimitte.

\subsection{Aurei coloris scriptura in cartis, in marmore, in vitro, ut videantur de auro.}
\paragraph{}
Aurum erugine misces, teris, unguis; vel argentum vivum teris cum luteo mulieris, et ungis.

\subsection{Deauratio vitrorum, in calamo, et in ere.}
\paragraph{}
Jacinctinum colorem (quo pictores utuntur) mitte in sal, et commove, donec solvatur. Deinde unge aurum, et infice, usque quater.

\subsection{Aureum colorem habere quem voles.}
\paragraph{}
Auripigmenti, cui aureus color sit, $\svgaab$. 40. argenti vivi $\epsilon$. 15. cricolle z. 10. vini peregrini z. 20. elquimatis 1. molipdini C 2. sulfuris z. 2. eris Gallici limature z. 20. come cucumeris percandidi z. 4. Omnia contundis, et cribellas spisso cribello, et proicis quod lignosum remanserit; deinde macera albore ovorum 50. et permitte siccari. Rursus contunde, et macera gummini liquido et ovis, donec fiat mellis pinguedo: et tunc dimitte in formam, et patere triduo. Postea tollis, et habebis signum aureum melius veritate; et, ut nichil honerosius sit dictum, absconde confectionem.

\subsection{Auri solutio ad picturam.}
\paragraph{}
Auripigmentum, et sepie ossa, et eris florem, equis portionibus; et sandaracam et spumam argenti, que aurei coloris est, ovorum vitellos, equis portionibus, dragantum et fel caprinum cum prioribus triveris, et, cum miscueris liquore solo fellis, utere in lamina, et signa, percurres species que unguentur. Fac autem tusus ferro limato splendido, non aspero. Inunges thure igni usto: mirabiliter facit.

\subsection{Aurum viride facere.}
\paragraph{}
Sumes auri partes 4. argenti partes 2. simul confla in unum; et, cum conflaveris in alia, et alia, quam vis masculinam imaginem facito, et habebis virilem colorem, et delectationem, et ostentationem non modicam, que prebet viventibus imaginum vivarum colorem. Si vero rubrum velis facere, admiscebis cypri partem 1. conflabis autem æs primum frequenter, donec fiet testeum, et conflabis cum predicta ponderatione. Si vero feminæ imaginem formare volueris, sume partem 1. et argenti pondera 4. et fit mixtum, demonstrans corpus femineum splendescens, cum fuerit extersum. Posthac inventum est ut fierent et nigre deorum imagines ex auro, et argento, et ere, et aliis permixtionibus. Mixtura autem et confectio in sequentibus demonstrabitur.

\subsection{Deauratio omnium, si veils deaurare, sive argenteum vas, sive ereum.}
\paragraph{}
Accipies laminam auream valde tenuem, forficabis minutatim, et mittis in mortarium: adicis argenti vivi modicum, et dimittis modico tempore. Adicis postea nitri aliquid, et aceti; teris pumice diligenter, donec fiat glutinis pinguedo, propter abundantiam argenti vivi. Hec autem mittis in pannum mundum, et exprimis, ex quo plurimum argenti vivi exiet. Tunc sumis vas, extergis pumice tenui, calefacis, et, dum refrigerat, deinde inungis de malagmate, et calefacis iterum vas, et rursus inungis, et igni dabis. Impinguaturque aurum solum. Cum itaque color placuerit, postmodum calefacies vas, et mittis in aqua melanterie, (\emph{id est}, aqua caligaliorum qua denigrantur coria;) et tunc teris. Si vero vas ereum deaures, post quam id exterseris, unguis alumine liquido, nam non recipiat malagma.

\subsection{In aureo vase nigrum compingere, ut putes inpisatum esse.}
\paragraph{}
Argenti, eris rubri, et plumbi, partes equales confla, et asperges sulfur vivum, et cum fuderis patere ut refrigerant: mittis in mortarium, teris, adicis acetum, et facis atramenti, de quo scribitur, pinguedinem, et scribe in auro et argento quod velis; et, cum refrigerat, calefacito, et erit inpistatum. Conflabis autem ita; carbonem sculpis, ita mitte argentum et es, et confla, et cum calefacis admisce plumbum, deinde sulphur; et, cum miscueris, defunde, et fac ut predictum est.

\subsection{(\emph{Opus ornatum.})}
\paragraph{}
Accipias lapidem androdomanta, teris et modicum misce de chrisocolla, et tangis ex ambobus; et mitte in fornacem, et expecta donec coquitur bene, et erit ornatum.

\subsection{(\emph{Argenti ornamenta.})}
\paragraph{}
Argentum confla, et cum arserit adice sulfur, et commove; et dimitte ut refrigeret. Tere bene; et ex eo inunge, et adicies nitrum, et oleum. Hec omnia uno veda sunt, priusquam aliquid fiant.

\subsection{Auri ornamenta, et eris opus ornatum prestabis.}
\paragraph{}
Auri optimi partes 3. chrisocolle Macedonice partes 4. eris floris partem 1. argenti optimi partes 3. sumis, et mittis medicamentum, et nitro rufo modico aspergis, et conflas levi igne: et, cum commiscueris in chrisocolla trita, auferes et operabis intente; et facies quale vis opus et signum.

\subsection{Deauratio stagnearum laminarum.}
\paragraph{}
Accipe laminas stagneas, contingis aceto et alumine, et conglutinabis glutine cartineo. Deinde sumis crocum et gluten purum, (\emph{id est}, perspicuum et limpidum); infunde in aqua cum aceto e limaturis: igne levi coquis, cum effulserit gluten inunge stagneas laminas, et apparebunt tibi auree. Vide autem ne elidrium ammisceas; si autem trita omnia feceris, noli mittere gluten, confirmabitur enim tibi opus; et si cum glutine sit concretum, ad auri scriptionem adicies elenusiam, ut perducas.

\subsection{Deauratio facilis.}
\paragraph{}
Sandaracam aurei coloris, et auripigmentum scissile, et dragantum, tere cum felle caprino et ovi interioribus, et ungue ante oleo novum opus; et, cum siccatum fuerit, habebit colorem primo auri similem.

\subsection{De conjunctione.}
\paragraph{}
Gummi comburis, et teris cum aqua inungis species, et ponis super levem lapidem, et in ignem mittis; ungue suprascripto medicamine, et conjungitur.

\subsection{Item.}
\paragraph{}
Cerusa et nitro Greco cum glutine taurino inunge partes abundanter.

\subsection{Ferrum auro conjungere.}
\paragraph{}
Ferrum calefacies leviter, et resinam similiter vaporans illinito, et inpone; et, cum panno alligaveris, purum circum ferrum adibe delicionem.

\subsection{}
\subsection{}
\subsection{}
\paragraph{}
Aurum crescere. Cipri z. 7. auripigmenti aurei coloris z. 6. Aurum facere. Cipri z. 6. Auri duplicatio. Argenti limature partem 1. chrisographiam Italicam de linitorio de lana, deauratio erit facilis, si hæc eadem feceris. [Inserted at the foot of the page, ``Auri solutio --- Fel taurinum mitte in vas novum, et dimitte in vase novo diebus 6. et postea liquabis, et cum extraxeris, conjunge et misce.'']

\subsection{(\emph{Quomodo scribitur de auro.})}
\paragraph{}
Sumis aurum, et mittis in vas novum, et adicis fel taurinum; obtera diebus 3. et, si videris dissolutum esse, effunde fel sensim, et adicis aliud fel cum aceto; atque inde rursum transfer in ereum vas purum; et cum tepefeceris vas purum ablue, et sicca. Ydrocollam adjunge, et sic scribe. Quod, si triduò solutum non fuerit, pluribus dimitte, et solvitur.

\subsection{(\emph{De vitri coctione.})}
\paragraph{}
Sumis vas vitreum, solidum, cavum, quod mortarii formam habeat, caraxas mirio de lapide, omnem ejus curvaturam subradens, in formam 10. littere. Pone cinerem, unde vitrum conflari debeat, in conflatorium, et asperges sanguine draconis crudo. Si autem crudum habere non voles, fac ex albumine ovorum, et succo visci medicamen, et in eo sanguine cinerem asperge, et sic excoque. Coctum, autem, cum in vitrum velis producere vas ad cujus debeat formam produci, item eodem sanguine delini, sicque factum, scias vitrum fragile in naturam fortioris metalli formari. Rumpere tamen hoc modo potes. Accipe sanguinem galli, et tere cum eo lapidem celidonium, admixta urina, cum toto lotio singularis non ammixto, addes etiam dragantum, et omnia simul mixta immitte in illud vas vitreum; et videbis artem, et ingenio vinci ingenium. Item, hoc ipsum, si in vas plumbeum vel stagneum miseris, solidatur ut es vel ferrum, tamen sine sanguine galli.

\subsection{Idem, (\emph{id est, auri scriptio}) quasi decus.}
\paragraph{}
Facias vas asperum, et quasi limatum; deinde mittis mel Atticum, deline aurum quod produxeris in avene modum, leviter circumfer manum, et, cum confricatur, leve fiat. Deinde postquam omne aurum, vel quanta volueris ejus pars, fuerit attrita, adice aquam melli et collue subtiliter, et liquato melle, invenies suhsidere aurum liquidum, et line. In hoc, igitur, quod residebit, adice sufficienter gluten taurinum, vel icciocollam,\footnote{Pro ``icthyocollam.''} vel gummi, quod te habere in vase propter usus oportet. Hinc itaque litteras scribe, et signa; oblinis et tecta subungis, et omne, quod volueris, aureum apparebit, vel cum unxeris, vel scripseris; et, cum siccaverit, dente frica, ut delinitum splendidum appareat. Hoc verò consilio, et ferrum, et plumbum, et eramentum solvis, et scribe, et inungis quod vis.

\subsection{(\emph{Item, auri scriptio.})}
\paragraph{}
Plumbum sepius liquefacis, et in aqua diffundis; et, cum hoc frequenter feceris in eadem aqua ubi plumbum refundis, exempto plumbo, confla aurum, et ibi dimitte. Frangitur enim in minutissimas partes; has tolle, et teris, et solvitur: huic adicis glutinis suprascripti quantum volueris; et utere ad que velis. Quod, verò, in ultimis arboris cavernis gummi invenitur, in his que deaurautur vitreis, si ante eo ungantur, bene retinet aurum.

\subsection{(\emph{Argenti confectio.})}
\paragraph{}
Misium ciprium, et sandaracam,\footnote{Sandaraca, \emph{i. e.} gumma vernicis.} et elidrium, equalibus portionibus accipis: et aqua in qua cocta sint folia sandaracis montani (\emph{id est}, papaveris agrestis, quod et ammonia dicitur) facis glutinis pinguedinem, et confla argentum, sed optimum; et, cum fuerit calidum, intinguis in predicta aqua.

\subsection{Argenti confectio.}
\paragraph{}
Eris cipri partes 2. argenti partem 1. sails ammoniaci \roundz. 4. aluminis scissi et liquaridie tantumdem. Omnia confla. Si autem volueris ex eo operari, expressionem cirii sumis, et uvam passam tritam, et mittis volaria in vas, et coquis plurimum, et tollis, et operaris ad ignem. Intinguis autem de medicamento, quod de coctione exit.

\subsection{(Eramentum candidum facere.)}
\paragraph{}
Sumis cuprum productile, quod cardarium (\emph{sic})\footnote{This word is written caldarium, but the \emph{l} is underdotted, and an \emph{r} written over.} dicitur, vel es ignitum, productum, facis laminas quibus substernis, et super aspergis cathmiam albam, tritam diligenter. (Nascitur in Dalmatia) qua utuntur erarii, et argilla oblinies fornacem diligenter, ita ne respiret, die una. Postea aperies, et, si bene habuerit, uteris; si non, secundò coquis cum cathmia, ut supra, quod si melius exierit cuprum caldarium permiscetur auro.

\subsection{Eramentum candidum facere.}
\paragraph{}
Cum conflari ceperit, adice auripigmentum, non procuratum, sed viride.

\subsection{Ad auri infectionem.}
\paragraph{}
Cum conflari ceperit, mitte auripigmentum quod curatum est.

\subsection{Album eramentum facere.}
\paragraph{}
Conflato eramine, adice auripigmentum non procuratum, et fit album, et quod teri possit; hoc lavabis aqua sepius, donec fiat purum; atque ex eo tollis, et invenies flavum. Deinde aqua ablues, et invenies eramentum ut sanguinem. Huic adicies argentum in fornace, et fit argentum simile corallio. Hujus partem 1. et auri partes 2. commisces, et facis miraculum.

\subsection{Plumbum similem argento facere.}
\paragraph{}
Plumbi minam 1. eris floris aliquid, corii mali Punici triti \roundz. 4. in cacabum mittis; obline cum creta, et sine, donec confletur.

\subsection{Chrisocolle confectio.}
\paragraph{}
Plumbi \roundz. 1. eris \roundz. 4. uná conflabis, admiscens Samiam terram, et salem, et alumen liquidum, et sine dissolvi; simulque ut ceperit purgato medicamine. Si autem est igni productile (\emph{id est}, caldarium) aceto tingue, et diffunde in quam volueris formam. Fiet enim non minime album.

\subsection{Eris commutatio.}
\paragraph{}
Sumis salis minas sex, limature, vel rasure, minas 4. Misce limaturam cum trito sale in vase, aspergens aceto, et dimitte diebus 3. et invenies viride factum.

\subsection{Item.}
\paragraph{}
Spuma argenti \roundz. 4. cum stercore columbino, et aceto teris. Scribe grafio calefactum.

\subsection{Argenteis litteris scribere.}
\paragraph{}
Argentum vivum et tornaturam de stanno tere; diffundes acetum liquidum, et, cum triveris, fac pinguedinem ex qua scribere possis.

\subsection{Ex ere argentum, vel elidrium, vel aurum facere.}
\paragraph{}
Sumis eris partes 4. et argenti partem 1., confla, et adice auripigmentum non ustum, (hoc est, curatum,) partes 4. argenti partem 1. et, cum valde calefeceris, sinito ut refrigeret, et mitte in patinam, et argilla oblinitum. Tunc assa, donec fiat cerussa, et confla, et invenies argentum.

Si autem multum assaveris, fiet electrum. Cui si partem 1. auri addideris, fit aurum optimum.

\subsection{Quo modo oportet argentum nigrum candidum facere.}
\paragraph{}
Sumis minii solidos 2. ex quo fit argentum vivum, quod aqua calida lavatur, et siccatum in usus venit. Uná conflabis, admiscens modicum calcidrium, et argentum, et plumbum. Accipiunt autem omnia plumbum conflatur, et sic invenitur splendidius assignatum.

\subsection{Argenti mixtura.}
\paragraph{}
Fel vituli teris, et mirram muliebrem, et semen rute tertiam partem, et rubeæ masculinæ similiter tertiam partem; deinde infunde tertiam partem, deinde infunde in argentum, et pone in summum fornacis, et diffundis, donec calefiat; et tunc adicies intus in fornacem.

Eris cipri partem 1. stagni partem 1. unà conflentur in antabram (Est autem antabra forma pecuniaria).

Argenti partes 2. stanni purgati 3. partes. Purgatur autem stannum sic; pice et bitumine admixto conflatur, eris albi dimidiam partem confla simul; deinde tolle et tere, et fac quod vis.

\subsection{Argentea vasa tergere sine abusia.}
\paragraph{}
Sumis lanam sordidam, infusam aqua salsa viscida, et extergis: abluis aqua frigida, et patieris siccare. Hinc extergis, nullam aposiam facit.

Nitrum russum, alumen non satis purum, germina porrorum, marinos (? marmos) succo teris, cum uve lupine succo, et inungc argentum solida penna; calefac in quo et argentum vivum erat.

Nitrum russum et alumen simul confla; deinde teris cum aqua, unges argentum et calefacis.

Eris cupri minam dimidiam sume, et stanni minam 1. magnesie \roundz. 8., piritis lapidis triti \roundz. 20., conflabis, et immitte stannum; deinde parum novissime argentum vivum, et cum ferro commoveris; infunde in columnellas.

Minium cum alumine tere, defunde acetum album, et facies pinguedinem ceroti. Cumque frequenter terseris, patere per totam noctem ita esse.

\subsection{Argentum aureo colore apparere.}
\paragraph{}
Minium, alumen liquidum, cimoliam, equalia. Hec fundes in aquam marinam, cumque frequenter calefeceris, argentum in eo tingue.

\subsection{Candidi confectio.}
\paragraph{}
Candidi limature \roundz. 4. recipe, et argentum vivum; et, cum non innovaveris, asperge alumine rotundo, et confla.

\subsection{Liquidi argenti confectio, per quod quis aurum de-argentet.}
\paragraph{}
Argenti limature \roundz. 4. mellis Attici \roundz. 15. resine liquide \roundz. 5. eris usti \roundz. 2. liquefac in sale. Deinde repone in pixide.

Stanni purgati libram 1. argenti vivi \roundz. 1. Brundisini speculi tusi et cribellati \roundz. 1.; admisceatur argentum vivum stanno, cetera vero singulatim conflentur, et misceantur; et, cum factum fuerit liquidum, calefacis frequenter eramen, et, cum resederit et unxeris, intinguis in illum liquorem, et sicuti aque stagnantur summi os vasculi calefacies, et non exiet.

Plumbi usti aliquantulum teris in mortario, admiscens sulfuris modicum, et facis glutinis pinguedinem cum aceto: scribe vasa argentea, et, cum siccaverit, calefacis, et nunquam delebitur.

Eris partem 1. stagni partes 5. plumbi partem 1. teris omnia simul. Crisocollam, argentum vivum, Samiam terram rufam, mel, in unum conteris, et inungis vas ereum, et assabis.

\subsection{Argentum colorem safirinum.}
\paragraph{}
Est autem potio permixtionis hujus; alumen scissile in aqua decoquis, adicis argentum vivum, et lapidem assum Bisaniten partem 1. laminæ argenteæ partem 1. Hec fiunt ut lotum formare possis quod vis, et patere siccari triduo, neque ante formaveris, ne narres quid est, nam exterminabitur, arenosum enim fiet: calefac, autem, non contingens nisi semel; si verò bis volueris, perdis mixturam: formabis signa non dissona a viriditate saphirain (? saphiram). Si vero volueris eadem deaurare, chrisitem, et crocum gummi liquido permisce.

\subsection{Ut argentum, et es, auri colorem excipiat.}
\paragraph{}
Auri partem 1. plumbi partem 1. hec simul confla, deinde limabis, teris in mortario duro Thebaico, aquam et nitrum adiciens: post infundes in vas plumbeum; deinde intingis vas, et mittis in fornacem, et calefacis donec auri colorem habeat.

Accipies stagni \roundz. 2. stanni \roundz. 2. confla, et cum permixta fuerint utraque peribunt, et fit fragilis materia. Hanc teris in mortario duro, adicies gummi et blude siccum omni modo; et, cum siccaverit, deline, ut subjectum est. Si autem aurei coloris velis videre, crocum tritum cum glutine puro permisces; quod delinias scribere, cum scripseris et siccaverit, delinimento tere.

\subsection{Argenti inscriptio Italica.}
\paragraph{}
Sumis laminam argenteam; teris, ut infra scriptum est, cum sale, vel nitro. Deinde ablues aqua, adice fel taurinum, et, conterens in vas vitreum, repones. Scribe penna, aut pincello pictorio; cum siccaverit, delinies.

Sumis laminas argenteas, teris cum argento vivo, et stilbada, et alumine, et gummi, et aceto sic scribe. Ut perseveret scriptura, misce omni colori aqua fabriferiarii (? fabriferrarii), et aluminis tantum, et sic scribe, cum ante diligenter exterseris.

Eris florem, eruginem, sulfur, in aceto dimittis: scribe adjecto gummo.

Fricabis argentum in mortario basanitis lapidis, cum aqua modica, et liquabis aqua, et siccabit argentum tritum, ydrocollam recipiens: scribe hoc ipso, et in auro faciens similiter utere.

Argenti \roundz. 1. confla, et, cum solutum fuerit, adicies stanni puri \roundz. 3. et defundens, patere ut refrigescat. Deinde lima et tere, et scribe quod volueris.

Stanni partes 2. plumbi partes 3. simul confla; et cum feceris, deline, lima, et tere: deinde adice ydrocollam, et deline.

\subsection{De argento vivo scribere.}
\paragraph{}
Accipe argentum vivum, et mittis in pultarium; adicies calcis vivæ modicum, et aluminis liquidi modicum, et acceti acerrimi aliquantum, quod commoves donec admodum simile fiat; depone de foco, tere, et mitte in pannum; exprime, et descendet argentum vivum; et ammisce vitrocollam, (\emph{sic}) et scribe.

Eris cipri \roundz. 2. stanni et magnesiæ albæ \roundz. 1. spumæ argenti \roundz. 4.

Sumis stannum candidum et tenue, purgabis quater, et argenti partem 1. conflabis; et, cum conflaveris, tere diligenter, et fabrica quæ velis, sive pocula, vel quicquid tibi visum fuerit. Erit enim ut primum argentum, qui etiam artifices fiant.

Sumis stannum purum, ut nosti, stanni \roundz. 1. magnesiæ scripula 3. scumatos mollis purgas, limas ex eo scripulis 3. argenti vivi scripula 3. Postea sumis limaturam ferri, teris in mortario, donec fiat lanugo; deinde mittis corpus magnesiæ, et hoc teris cum eo diligenter, donec fiat malagma, et hoc mittis in fornacem, conflabis, deinde mittis, et stannum, in aliam fornacem. Hanc scripturam discito.

\subsection{Eris usum, argento similem, facere.}
\paragraph{}
Chrisocolle, ceruissæ terrene, argenti vivi, singuloruin \roundz. 2. teris, et infundis mellis boni quod sufficiat, et calefacis; cum, quod vis, inungis, leviter subice ignem de lignis leucinis (\emph{id est}, violaceis), cum ante terseris vas.

Argentum vivum et sulfur inunge, et erit miseos cipri cipri (\emph{sic}) minam 1. miseos assi libram 1. aluminis sordidi \roundz. 14. Accipies in manu salem tenuem, quem in manu feres donec nigeat. Staterem, diligenter tersum, defrica cum sale, et, ubi factus erit argenteus, de eneo statere denarium duci foliis vitteis obvolve, et patere tota nocte ita esse: sequenti die tolle et utere.

\subsection{(\emph{Litteras virides facere.})}
\paragraph{}
In ere, vel ligno, vel lapide, aut in quocunque vis, litteras virides facere perpetuas.

Accipias naxias limaturam, de cute tonsuram, et tere similiter limaturam agaciam, et alumen, et de lacu aqua (\emph{id est}, pluvialem) et ordei seminati herbam: omnia simul terens, scribe, et erunt virides litteræ.

Extergis es, et pumicas diligenter, et dimittis in sole; postea, accipies citri partem 1. et de malo Punico ligna, concide diligenter, et tolle ex ipsis mensuram convenientem cum nitro et aqua et sale faciens. Mitte es, et dimitte illic diebus 5., et, cum intinctum fuerit, inunge cerotum.

Columbaris spinæ suce (\emph{sic}) partem 1. cum alumine teris; recipis tusum diligenter, et dimitte biduo; et, cum tuleris, inunge et dimitte die 1. et tunc exterge.

Accipies argenti \roundz. 1. eris \roundz. 1. plumbi dimidiam; confla simul. Cum autem ungere velis, appone gluten eri, et calefactum ferrum adjunge glutini, quod est appositum eri, et sic ornabis. Sin autem laminam laminæ conjungis.

\subsection{Colorem viridem facere.}
\paragraph{}
Cuprum, productum in laminas, line cum melle, vel spuma mellis cocti; et suppone spatulas ligneas in vase, quibus suffundas lotium hominis; et stet per dies 14. coopertum.

Auricalcum productum line cum melle cocto, et infunde ei urinam, et acetum, pari mensura, et sit opertum dies 14. et erit quasi lazur.

\subsection{Indicum colorem facere.}
\paragraph{}
Succum de ba(c)cis ebuli collige, et diligenter sicca ad solem; de hoc quod remanserit fac pastillos, cum parvo aceti, et vini, et utere.

\subsection{Collam Grecam facere.}
\paragraph{}
De vernice fac farinam tritam in marmore, et cribratam; et mitte in ollam rudem, strictam diligenter operculo clausam, ita ut in medio operculo sit parvum foramen, et in ipso foramine stilus ferreus; et pone in laminis ferreis super fornacem aurificis, que prius debet incendi. Deinde suppone ligna arida minutissime concisa, et, statim ut incaluerit, liquescit. Extrahe stilum ferreum, et guttulam pone super unguem; et, si liquida apparuerit, subtrahe focum, et infunde olei de semine lini expressi 2. partes, ad 1. partem vernicis, et iterum, suppositis lignis, coque ad horam parvulam, et utere. Si verò sit de grano masticæ, liquescit tardius.

\subsection{Inauratio lapidis, vel ligni, vel vitri.}
\paragraph{}
Qui autem vitrum inaurat, tollat partem de glutine picis, et partem de gumma amigdalæ; et, miscens, coque, et ungue ipsum vas, et concide subtiliter petalam auri, pone secundum similitudinem quam vis facere. Similiter et lapidem lavans aqua, et lignum; et, dum siccaverit glutinatio, cum emathite lapide, aut cum ferro, defrica.

\subsection{Ad colorandum aurum.}
\paragraph{}
Sume atramentum, et assa, ut scis, et aliud, tantundem salis, et distempera cum vino rubeo, in eneo vase non nimis rarum, et inde line aurum; et pone in fornace, et tamdiu calefac ut nigrum fiat, et extrahe.

\subsection{De dispositione Fabrice.}
\paragraph{}
Dispositio fabrice de pontibus, vel quibus mensuris oporteat edificia disponere, vel quibus mensuris in altitudinem elevare, secundum modum fabrice.

Si in altitudinem 4. staturis fuerit, fabricam unius staturæ altitudine oportet esse fundamentum. Si vero tribus staturis altitude, usque ad bifurcum erit fundamentum. Si autem unius staturæ altitudo, usque ad geniculum fundamentum. Si statura 4. cubitorum,\footnote{Something seems to be omitted here.} si vero 3. usque addida, si duorum, usque ad furcam. Si in lignis opertum in altitudine fuerit. Si voltile fuerit, quantum in altitudinem, tantum et in fundamentum debes cavare; ita videlicet altitudinem mensurari oportet, ut tantum parietis absque camera mensuretur. Si autem durus fuerit locus et montuosus, cubito minus per staturam pones fundamentum. Si mollis locus fuerit, sicut supra diximus, edifices. Si vero petrosus fuerit locus, non crede petris, sed cava, sicut oportet, ne pondere nimio deprimatur, et subsidat fabrica.

\subsection{De Fabrica in aqua.}
\paragraph{}
Si fabricam in aqua necesse fuerit erigere, facis arcam triangulam, et picas eam foris cum sepo et pice, ut non in eam intret aqua, et solvat ipsam calcem, et eos qui laborant intus; et positam arcam inter 4. naves, constitues in loco ubi necesse fuerit: et oneratas ipsas naves, ut non moveantur in aqua, et tunc impones lapides ad fabricandum. Temperatio autem calcis talis fiat. Mittis arene partem 1. et calcis 2. et tunc operaris. Ipsa autem arca habeat 1. cubitum super aquam.

\subsection{De Multa.}
\paragraph{}
Multa quoque debet ita confici. Mittis calcis partem 1. arenæ partes 3. vel 4. teste tuse tertiam, pulveris palearum sextam partem, aque verò congium 1. olei porcini sextaria 2. et requiescat ebdomada 1.; si plus dimiseris, melior fiet. Assidue autem infundatur secundum mensuram quam indiget, et conficiatur, et tunc operare.

\subsection{De licamonia.}
\paragraph{}
Licamonia, alumen Egiptiacum, fresa solis 3. nitri z. 1.

\subsection{De compositione cinnabarin.}
\paragraph{}
Tolles ydroargiris mundi partes 2. sulfuris vivi partem 1. et mitte in ampullam, sine fumo, et lento igni, decoquens, facies cinnabarin, et lava utiliter.

\subsection{De compositione iarin.}
\paragraph{}
Tolles petalam mundissimam de eramine, et suspende super acetum acerrimum: pone ad solem immobiliter per 14. dies: aperies, et tolles ipsam petalam, colliges florem; facies iarin mundissimum.

\subsection{De compositione psimithi.}
\paragraph{}
Tolles plumbum, et facies petalam, suspende super acetum; colliges ipsum florem, et lavas bene, donec mundus fiat, et facies psimithin. Posthac, tolles de cinnabarin partem 1. de iarin partem dimidiam, et de psimithin partem dimidiam, et mittis in mortarium marmoreum, et teris bene. Post tritionem autem, mittis ex aqua, ubi coquitur icciocollon, et fiet pigmentum pandium.

\subsection{Compositio Lazurin.}
\paragraph{}
Folia floris violæ colliges, et in mortario mundo teres bene, et mitte saponem ex axungia, sine calce: mundum facies, squama cum aqua tepida, et solve saponis in libram aque $\div$ 1. et fricas subtiliter ipsum saponem cum aqua, et dimittes refrigerari. Et posthac, mittis ipsam commixtionem in tritura florum, et mittis illam in vas vitreum, in quo possis mittere, et repones ibi: et post aliquod tempus commisces, et permove illud cotidie in die semel, usque ad imam ebdomadam. Posthac 3. dimittes, et duos permoves, donec decoques. Postea tolles lilium fuscum majus, quod est porfirius, quod habet folia veluti cultellus; similiter et ipsum defricas in mortario utiliter, et dimittes sine sapone, mittens aquam; et postea, ex violæ compositione libras 2. de lilio majore fusco libram 1. aluminis Egyptii spumati, si forte est inspumatum, saponem, aut, si debile est, et crudum, mittes \roundz. 2. urinæ spumatæ, libras 2. et lento igni decoques, per horas 6. et si multum viride est, mittes urinam; si vero multum venetum, plus alumen mittas.

Si autem lazurin viscidum ex albo lilio domestico mittas quod sufficit, et decoques. Gustum autem coctionis in ligno unde movetur considera. Calidum enim non demonstrat, frigidum vero ostendit colorem. Coctio autem lento igni debet esse. Aqua enim cum coquitur expendet, quare et ipsa commixta debet esse cum sapone, secundum commixtionem quam prediximus, et ita coctioni addi. Quando autem de fornace tollis cacabum testeum bituminatum, aqua vult superare commixtionem.

Alumen verò in forti sapone expumato sit. Mitte alumen in (aqua)\footnote{Not in MS.} tepida, dimitte residere alumen visum, et effunde ipsam aquam tepidam, et ita expuma alumen. Ipsam autem commixtionem sic defrica, ut in pulverem redigatur, et non sint species in lazurin. Nam post coctionem ipsam comixtio terenda est, et movenda, et in umbra desiccanda in ipso cacabo; et postea ad solem eicienda, et torrenda, ut fiat lazurin.

\subsection{Ut pictura aqua deleri non possit.}
\paragraph{}
Oleo, quod apellatur cicinum, super picturam ad solem perunge, et ita constringitur, ut nunquam deleri possit.

\subsection{Confectio pandii.}
\paragraph{}
Folia florum papaveris nigri collecta repones in pinnato novo; et operiens pones ad solem 1. die: et, dum marcida fuerint, tolles aquam in qua icciocolla decoquitur, et mitte in ipsam folia florum, et teres utiliter, miscens cum modico cinnabarin, ex exiet color pandius.

\subsection{Confectio ficarin.}
\paragraph{}
Tolles laccam mundam, et teres munditer, et coques eam in urina spumata, et iottam, que exiet, refundes in vas vitreum. Postea tolles siccum et bene tritum pulverem de lilio albo domestico. Si autem infuscaverit, non mittas, ne fuscum fiat. Istud tritum in libras 2. de irinico flore, et ipsum tritum libras 3. pulveris duarum commixtum unicuique coctioni lacce $\svgaaa$ 1. de alumine Egyptiaco commisce, et teres bene: mittes in vase testeo novo bituminato, et pones ut tepescat modicum, et non comburatur: posthac, commisce coctiones ad laccam, et facies bullire. Tolle de fornace, sicca ad solem.

\subsection{Deauratio in ligno, vel in panno.}
\paragraph{}
Si in ligno debet fieri deauratio, gumma amigdalæ infusa die una; postea teres utiliter ipsam gummam cum aqua, et addito croco, quod sufficiat, tingue in ipsam aquam cum gumma et tepefacito omnia lento igni, operare in ligno quando opus est. In pannis vero, vel in parietibus, tolles albuginem ovi subtile, et addito croco, quod sufficiat, tingue, et commixta ac trita repones in vase vitreo.

Item, lineleon. $\svgaaa$ 1. gummæ infusæ $\svgaaa$ 1. crocum, quod sufficiat, commisce: cum aqua decoques.

\emph{Rubric.} Ista tria capitula sequentia ubi necesse fuerit in exauratione petalarum operare.

\subsection{Compositio lineleon.}
\paragraph{}
Lineleon libras 2. gumme \roundz. 1. resine pinii $\svgaaa$ 1. Omnia hæc trita decoques in vase testeo. Lineleon libras 2. gumme \roundz. 2. resine \roundz. 1. croci solidos 2. Ista tria commisce sicut superius.

\subsection{De lineleon in exauratione.}
\paragraph{}
Operatio sternituræ exaurationis, si super pellem crudam immobile tinctam, aut ex psimithin, aut ex aliquo colore sternitum est, ista a chrisopetalam reponuntur; et, post desiccationem, desuper lineleon perunge, ex commixtione quam supra docuimus, ubi croco componi dicimus.

\subsection{De inductione exaurationis petalarum.}
\paragraph{}
Petala fiat de stanno: fiat autem sic. Solves bene ipsum stannum, et fundes paulatim in marmore, et facies petala subtilia, veluti ex auro: et pones sicut chrisopetala, ut supra docuimus, et decoques herbam celidoniam, et, ex ipsa coctione colata, mittens $\svgaaa$ 3. croci solidos 3. auripigmenti solidum 1.

\subsection{Tinctio stagneæ petalæ.}
\paragraph{}
Tolle croci mundi $\svgaaa$ 1. pigmenti optimi fissi $\svgaaa$ 2. mitte gummi dimidiam, et lineleon $\svgaaa$ dimidiam, et aquam pluvialem, aut dulcem, commisce, ut bulliant simul. Commisce confectiones, terens bene, tollesque cum spongia, unge ipsam petalam; et, cum desiccaverit, secundo unge, et desiccatam cum onichino defrica, ut splendeat.

\subsection{Confectio crisocollæ.}
\paragraph{}
Caucucecaumenum $\svgaaa$ 1. saponis olei solidos 3. calcitirii (\emph{sic}) solidum 1. Ista commisce, primum terens caucucecaumenum utiliter ad pulverem, et calcitarin (\emph{sic}) semotim, et commisce cum sapone et aqua, quantum necesse fuerit, ad ipsum chrisocollen.

\subsection{Aliud crisocollon.}
\paragraph{}
Caucucecaumeni libram 1. aluminis, sol. 2.

\subsection{Item.}
\paragraph{}
Aurum commixtum cum argento vivo mittitur in calidum, donec ardeat ipsum argentum vivum. Postea, tolles aurum et teres in mortario, donec fiat pulvis; et commisce illud cum sapone ex oleo, quantum sufficiat ad commixtionem chrisocolli.

\subsection{Item.}
\paragraph{}
Argenti partes 2. eraminis partem 1.

\subsection{Item.}
\paragraph{}
Argentum mixtum cum argento vivo pones in ignem, donec siccetur ipsum argentum vivum. Deinde tere illud donec fiat pulvis: commisce cum sapone et aqua, quantum satis est.

\subsection{Eramenti gluten.}
\paragraph{}
Eraminis libram 1. plumbi libras 2. commisces, solves primum eramen, deinde mittes plumbum, et commisce in unum.

\subsection{Gluten de ligno vel osse.}
\paragraph{}
Glutinatio ligni in aqua. Icciocollon \roundz. 1. lactis fici \roundz. 1. titimali lac \roundz. 1. ista comisce in aqua, decoque. Est autem gluten ad sculpta ligna: si lignum in lignum, unum ex supradictis 3. Si autem ossa in lignis, casei gluten \roundz. 1. commixtum cum icciocollon \roundz. 2. decoques in unum, et gluten calidum, calefacis modicum ipsa ossa, et sic glutinas.

\subsection{De metallo auri ad coctionem.}
\paragraph{}
Indicamus vobis quomodo fieri possit aurum de pinguedine metalli. Dum ipsum metallum inventum fuerit, facito vas quod ipso metallo recipere possit libras 20. et postea mitte cum ipso vase in fornacem, et sufla ignem ab hora prima usque 6. Postea vero in pinguedine metalli mittendum est coralli libræ 2. amoniacum fundatum, calcum, aurum, libræ 2. sal bedica libræ 2. cera alba quantum opus fuerit; unctum libræ 2. tartarum libra 1. coctum de omni pigmento singulos per se intrantes. Que vos legistis omnia probata habemus, quia tria metalla ad aurum coquendum pertinent. Et aliud metallum indicamus vobis coquendum, sed plus disculum erit, quam metallum auri. Qui ipsum coquere voluerit, sicut ros erit odore, et in ipso vase, ubi coctum fuerit, mittito primam cocturam pice ad aste (? arte) medietatem libræ. Alia vero fersurura vitrum mastallo, tertia coctura stanni libras 2. qui ipsum metallum ad opera salva perducat, et, dum coctum fuerit, istud, quod in ipso metallo mixtum fuerit, ad pulverem vertitur, quia probatum est.

\subsection{De metallo argenti.}
\paragraph{}
Prassinus, terra est viridis, ex quo metallo manat argentum. Nascitur autem et ipsa terra in locis petrosis ubi inveniuntur multa metalla diversis coloribus. Ista petra trita albas habet venas, decoctas exeunt nigra, sic autem probatur. Comminuta post coctionem intus ut argenti colores ostendit: iste lapis est de quo exiet argentum. Tolle ex ipso metallo, fornace enim sicut superius primæ cathmiæ, et mittis ipsum metallum in canciacami, et imple carbonibus; et sic decoques, et fundes die una, et dimittis refrigerare in ipso loco; et posthac tolles ipsam massam, et comminues minutatim, remittes in ipso camino, sicut prius, et cum ipso plumbum femininum. In centum libras massæ plumbi 15. et coque sicut prius per dies 3. Posthac eice ipsam massam, et comminue; mittes in calida vel in tanida, et confla per 2. horas.

\subsection{De lapide adamante.}
\paragraph{}
Lapis adamas nascitur ex cathmia, et auri coctione, in prima coctione massæ. Post primam cocturam, dum confringis massam; omnis enim massa confringitur leviter; iis autem remanet, alius parvus, alius magnus, cui ferrum non dominatur, nec aliud quid aliorum lapidum. Ipse autem omnibus prevalet; solo vincitur plumbo, et hec est potentia plumbi.

Tolles plumbum femineum, facile et molle, et solves, et jactes ibi ipsum adamantem, partem quam volueris subtiliare; et lento igni succendes plumbum; et, dum ceperit subtiliari, continuo cum mordace tolle, et in sapone ex oleo operiens, leniter ac mundissime, eo quod sit debilis. Est enim fragilis plus quam vitrum, et mollis plus quam plumbum, eo quod solvatur in plumbo; deinde tolle illum de sapone, et in cote aquaria exacuas cum ipso sapone, quantum volueris subtiliare, et mitte in ignem magnum diligenter, et excandeat per horas 2. aut 3. donec candeat totus sufficienter. Postea tolle et lava, et exiet adamas cui ignis non dominatur, nec feriendus dissipatur, et laborans non curtatur; per quem omnia que volueris operari potes.

\subsection{De conchilio tinctio porfirii.}
\paragraph{}
Conchilium nascitur in omni mari, plus quam in insularum locum. Conchula est, et habet in se locum sanguinis, et sanguis rubens porfirizontas, ex quo porfira tinguitur. Colligitur autem sic. Tolle conchilium, et collige ipsum sanguinem cum carnibus, et tolle muriam de mari, et compones in vas, et dimittes.

\subsection{De porfirio citrino.}
\paragraph{}
Tolles alumen Alexandrinum; tere utiliter, et pone in gabadam, et mitte super caldam bullientem; permove diutius, et dimitte residere. Posthac cola ipsam caldam, et exagita; et mittes aliam caldam, et agita; et impone ibi quod habes tinguere, et cooperi, et dimitte 2. dies. Post hoc commove, et fac quod visum est sursum, et dimitte alios 3. dies, et post hoc exagita similiter et dimitte alios 8. dies, et exagita die infra duabus vicibus. Deinde tolle, et mitte aliud alumen, et facies exinde aliam tinctionem, et in eam mitte, et post tolle urinam mundam ex vino bono, et viris sanis, tollesque ipsam urinam, et spuma semel, et post hoc mitte in cacabum ereum; et tolle ipsum conchilium, et lava leviter semel in aqua. Post hoc teres, pone in pannos raros, delava in ipsa urina cacabi. Post hoc tolles de sanguine porcino, et de farna (? farina), et ipsum bene similiter lavans, sanguinem autem porcinum garidum libra conchilii $\svgaaa$ 3. de sanguine porci. Post hoc lavas semel modicum, et defrica; mitte in cacabum, et fac bullire secundo, et tertio, sub eodem modo libram enim tinctionis, libram conchilii cum sanguine; id est, 9. uncia conchilii, et 3. sanguinis porcini.

\subsection{De oxiporfironta aporodinis.}
\paragraph{}
Oxiporfironto aporodinis tolles trium cacaborum coctiones; et mitte in unum in eadem coctione quicquid volueris, tantum ex alumine. Si autem volueris plus munditer tinguere, mitte in unum cacabum sicut primum. Fiet enim et tercia tinctio eodem modo.

\subsection{De porfiro citrino.}
\paragraph{}
Prius enim tinguitur citrinum, et posthac intrat in tinctione, ubi tinguitur porfirus.

\subsection{De Crisorantida, de auri sparsione.}
\paragraph{}
Chrisopandium pulverem auri triti, sicut superius diximus, cum desiccatione argenti vivi, \emph{id est}, pulveris auri 2. et iarin partem 1. commisce cum compositione daufiras, et dispone inde quod volueris.

\subsection{De argirosantista, de argenti sparsione.}
\paragraph{}
Argentum mundum commisce cum argento vivo. Post hoc ponis in ignem, et desiccas cum ipso argento vivo: deinde tolle ipsum argentum, et tere, donec fiat pulvis: commisce cum impositione daufira, et dispone ubi volueris. Tolle argentum mundum, et commisce cum argento vivo, sicut supra diximus; deinde mitte in caliculo, et depone in ignem donec eiciat argentum vivum. Post hoc tolle argenti partes 2. et iarin partem 1. et commisce ex compositione daufira, et dispone.

\subsection{De Smiria petra.}
\paragraph{}
Petra, quæ dicitur Smirias, asper et indomitus est, omnia terens; cum quo lapides, gemmæque limantur.

\subsection{De terra limia.}
\paragraph{}
Terra que vocatur limia, que est alba sub porfira, nascitur in petrosis locis; et his signis cognosces eam. Cum infusa fuerit, bulliet, et sonum dat stridoris: alumen vero viride, et porphirium, omnia tinget, extra berillum et onichinum. Terra nigra vocatur, eo quod est fusca. Nascitur enim in Egypto, et Africa, et in Evilath, et in Italia. Nascitur enim in humidis locis, et in vallibus. Ex ea tinguitur roseum, commixta cum aceto, et cocta, reddet colorem; et post hoc revertitur in coccum.

\subsection{De lapide focario.}
\paragraph{}
Lapis, qui dicitur focarius, ex quo eramen tingitur, nascitur in omnibus locis. Est et alius similis dum percutitur emittit scintillas raras et magnas; et est rubeus et igneus, colorem habens eramenti; et dum in ignem missus fuerit, ut probetur, incenditur, et colorem non mutat. Collectum autem et contritum minutatim, collecta massa modica, et cum letamine bovino, aut caprino, et palea cooperto, incendes per dies et noctes, donec consumetur ipsum letamen in fornace. Ipsum autem potes coquere, et eramen, et plumbum; et, postquam refrigescet, colligitur ipse lapis, qui jam coctus est, et in prima pensione, pensas libras 300. in prima coctura. In quam mittis carbonum cofinos 18. fasces de teda; et dum ingressi fuerint magistri ad opera, et incensa omnia dimittes stare ea, et refrigerare, ut non currat, si est plumbum, aut ferrum, aut cetera metalla, eo quod piger fiat. Et, dum refrigescit, comminue minutatim, et in fornace conflatur, et operaris ex eo.

\subsection{De lapide fisso.}
\paragraph{}
Lapis fissus nascitur in Capadocia, Asia, Hiberia, et in Italia. Est enim fuscus et fortis. Dum comminutus fuerit, invenies in eo venas albas; et, cum incensus fuerit, fiet rubeus; quem Alexandrini vocant ``cathmiam,'' eo quod conflet vitrum. Nascitur autem in altis locis, et ventosis. Est enim lapis crepidinosus.

\subsection{Quomodo fiat cera-marmor ex gagate, et de lapide tracho.}
\paragraph{}
Lapis gagatis similis coloris auripigmenti, non enim sic multum viridis; qui, dum rumpitur, ignem emittit, et finditur in laminas; propter quod Alexandrini vocant eam ``petram planam,'' ex qua fit cera-marmor. Quem, si pisas subtiliter, et mittis libram de ipso lapide, et 2. de aurocollon, et aquæ libras 5. et bis et tercio facis bullire, assidue commovens, et commiscens, fiet cera-marmor.

\subsection{De lapide tracho.}
\paragraph{}
Lapis trachias nascitur in universis locis; est enim viridis, fissus, fuscus; combustus, fiet albus; mittitur in cathmia in mundatione argenti.

\subsection{(\emph{De Caucucecaumenon.})}
\paragraph{}
Caucucecaumenon fit hoc modo. Ex eramento mundissimo facies petalas, mittasque in usitatum ipsas petalas, et sulphur vivum tritum; et iterum sterne petalas in cacabo: deinde super sparge sulfur, et ita facias donec impleas ipsum cacabum. Deinde, in fornace vitrarii posito cacabo, coque diebus 3. et, dum refrigescit, confringes eum minute. Adde alumen Asianum, secundum compositionem sulfuris; similiter cooperiatur ipse cacabus, et liniatur cum argilla, et ponatur secundum dispositionem prioris, et coquatur per dies 6. Ipse autem, dum confringitur, solvat caucucecaumenon ad gluten aureum.

\subsection{Compositio electri.}
\paragraph{}
Electrum fiet hoc modo. Pone 2. partes argenti, et eramenti tertiam, et auri tertiam (\emph{sic}); ita ut aurum et eramentum equis ponderibus fiant.

\subsection{Gluten auri ad fistulas.}
\paragraph{}
De chrisoclabo astici \roundz. 1. caucucecaumenon \roundz. 1. affronitri \roundz. 1. saponem ex oleo sine calce \roundz. 1. vitrioli solidi 2. aceti dimidiam \roundz. aque \roundz. 1. comminue, et commisce semotim eramen; cetera in unum commisce, et compone modicum, ut tepescat gluten de canilis chrisoclabi.

\subsection{Compositio litargiri ex plumbo.}
\paragraph{}
Litargirum, aliud ex plumbo, aliud autem ex argento, fit. Compositionem, quæ ex plumbo fit, sic compones. Plumbum impone in cacabo potius femineum et molle, et solves illud bene; deinde, cum solutum fuerit, pistillo ligneo frica plumbum incessanter: mitte cinerem similiter, et non desinas terendo quoad usque facias eum ut pulverem; et post laves eum aqua. Si autem stringi volueris eum, et fieri spissum, mitte in cacabo, vel in cabilo, vel in camiolos (? cannolos) cum oleo; et calefactum adimatur, et, dum refrigescit, franges caniclos.

\subsection{Alia compositio litargiri ex argento.}
\paragraph{}
Compositionem litargiri ex argento sic facies. Confla argentum, et illa sordidatio, que ex eo exit, trita cum oleo intrat, secundum priorem compositionem: plus autem incenditur propter fortitudinem argenti. Plumbi autem litargirum, ante quam solidetur, intrat cum aqua in bituminatione testea. Dum autem extrinxerit, ubi volueris, necessarium erit.

\subsection{Inauratio musii operis.}
\paragraph{}
Facies petalum vitreum spissum supra petalum eramentinum, ita ut incensum non cohereat. Posthac tolle petalum aureum super petalam vitri, et super petalam auri aliud pone ex vitro multum subtile; et mitte utrumque in fornacem, donec inchoet solvi petalum vitri; et sic eice ut refrigescat. Posthac frica faciem ejus in tabula plumbea smirutata, donec attenues faciem ejus, et coloras illud.

\subsection{De tabulis smirutatis.}
\paragraph{}
Facies tabulam de plumbo, et tolles smirram vivam; tere bene, et asperge tabulam totam, defricans semel vitrum, donec confringatur pulvis smirræ ad tabulam, et posthac operaris quod necesse est cum aqua.

\subsection{De coloratione musii.}
\paragraph{}
Ad colorationem autem tolle tabulam, et caraxa illam curtatim, et terens cretam argenti subtiliter, asperge tabulam, et defrica bene ipsum vitrum, donec coloretur.

Eraminis mundi limati partes 2. et aluminis Asiani, in mortario diligenter pisati et cribrati, partem 1. commisce in caliculo, et pone ad prunas, donec confletur, et commisceatur alumen cum eramine. Ipsam vero formam vasorum quam facere volueris prius munda urina, et sic funde eramen, quod in prima quidem, ac secunda inflatura, retinet colorem, in tertia perdet, cum limaveris, et batis illud continet colorem: si frangatur inutile erit.

Omnis gemma durioris naturæ, sicut jacinctus, et smaragldus, alemandina, carbunculus, fricantur ismiri lapide in plumbea tabula, usque dum formam accipiat, qualem illi cavator dare voluerit. Deinde fricatur in lotura ejusdem pulveris, in quo prius fricabatur, usque ad lenitatem. Splendificatur verò uno modo, sed jacinctus in pulvere lapidis igniarii combusti in lamina ciprea. Ceteræ vero splendificantur aut pulvere, facto de cimolia, aut ex fragmento testeo, quod fit ex vasis antiquis, in astula lignea tremuli, aut alni. At gemmæ mollioris naturæ fricantur, usque ad formam, in pulvere lapidis arenatii super plumbum, sicut sunt amethistus, cristallus, onichinus, jaspis, berillus; deinde fricantur usque ad lenitatem in lotura ejusdem pulveris. Splendificantur in pulvere lapidis igniarii combusti super laminam cipream. Vitrum verò fricandum est in lapide arenatio usque ad formam; deinde super plumbum, in pulvere minuto lapidis arenatii, usque ad lenitatem: deinde super hastulam ligneam, in confricatura testæ antiquæ fit, in cote cum aqua; novissime in cimolia splendorem accipit, et hoc in hastula lignea.

Frangitur ismiris lapis malleo super incudem durum in pulverem minutum; et fiat tabula de plumbo, quæ conficitur super scamnum ligneum; et pulvis illius lapidis super illam aspergitur, et in eo qualiscunque gemma formanda est; fricatur cum aqua, usque dum formam accipiat, quam ei cavator dare voluerit: deinde sumitur idem pulvis et lavatur: ex eo, quod minutissimum fuerit, super aliam laminam plumbeam ponitur, et in eo quælibet gemma fricatur usque ad lenitatem.

Sumitur lignum grossitudine minimi digiti, longitudine palmæ unius, et in ejus summitate pix calida, mixta cum tegula trita, ponitur; quæ mixtura duas partes pulveris de tegula, et tertiam picis, habere debet; in qua postea calefacta gemma, quæ fricanda est, sic ponenda est ut adhereat.

Eris pulvis, vel limatura, teritur cum aceto in eneo mortario, cum sale et alumine, usque ad mellis spissitudinem. Aliqui pro aceto aqua utuntur. Denique bene purgatum ferrum, et leviter calefactum, hac mixtura inungitur, et fricatur, donec colorem eris accipiat: tunc aqua abluitur, et teritur; et, sicut es, vel argentum, deauratur, et calefactum, recedente vivo argento, sicut mos est, ut splendorem accipiat ferro defricatur.

Alumen rotundum, et sal, quod gemma vocatur, et calcantum, ex aceto acerrimo, teruntur in ereo mortario; his ferrum purgatum cum ferula, vel alia qualibet molli hastula defricatur; et, cum eraminis colorem habuerit, extergitur, et deauratur, ac deinde, exfumigato argento vivo, aqua refrigeratum, usque ad splendorem ferro, valde plano et limpido, defricatur.

\subsection{Cathmie compositio.}
\paragraph{}
Eris mundi libra 1. calcitarin z. 2. affronitri z. 1. sulfuris z. 1. Hæc omnia mittes in calido, et solvantur in unum, et coquantur, donec comburatur eramen, et calcitarin, et levatur ea quæ remanet cathmia.

\subsection{Quianus ita fiet.}
\paragraph{}
Eris partem 1. plumbi partem 1. triti nitri z. 1. calcitarin z. 1. affronitri z. 1. commixta et combusta, commisces cum aceto, et repones ad solem; sicca et tere.

\subsection{Anfinus sic fit.}
\paragraph{}
Tolles plumbum molle, et solves in vase testeo forti, ut sufferat trituram: tolles pistillum, et mittes carbones cum cinere super plumbum, priusquam refrigescat; permoves illud cum pistello leniter et bene, donec attenues et subtilies ipsum plumbum: post hoc mittes in gabata lignea, et delavas. Deinde componas in cacabo novo cum sulfure, et decoques per dies 3.

\subsection{Pandii compositio.}
\paragraph{}
Psimithin partem 1. cinnabarin partem dimidiam, tere in mortario marmoreo bene: post contritionem autem, mitte ex aqua in qua coquitur icciocollon, et fiet pigmentum pandium.

\subsection{Alia.}
\paragraph{}
Iarin partes 2. cinnaberin partem 1. spimithin (\emph{sic}) partem 1. quiani partem 1. lulacin partem 1.

\subsection{Alia.}
\paragraph{}
Quianon solidos 3. spimithin (\emph{sic}) z. 1. nitri partem 1. calcitarin partem 1.

\subsection{Alia.}
\paragraph{}
Aluminis partem 1. sulfuris vivi partem 1. nitri partem 1.

\subsection{Tinctio vitri prassina.}
\paragraph{}
Tere vitrum bene, et de limaturis eris mundi z. 3. mitte in libram vitri, et decoques per dies 3.

\subsection{Alia.}
\paragraph{}
Vitro bene trito ad libram ejus adiciantur eris limature z. 2. aluminis Egyptii z. 1. et decoques per dies 3.

\subsection{Tinctio lactei coloris.}
\paragraph{}
In libram vitri mittas stanni z. 3. et decoques per dies 2.

\subsection{Tinctio sanguinea.}
\paragraph{}
In libram vitri mittas cinnabarin z. 3. et decoques per 2. dies.

\subsection{Tinctio rubea.}
\paragraph{}
In libram vitri mittas psimithin z. 2. et decoques per dies 6.

\subsection{Tinctio alithina absque igne.}
\paragraph{}
Tingues subtiles vitreas et ungues dracontea anamemigmemis, et fiet sic rubea.

\subsection{Minus tincta melini coloris.}
\paragraph{}
In libram vitri theaspis terra z. 2. et decoques dies 3.

\subsection{Rubeum.}
\paragraph{}
In libram vitri caucucecaumenon z. 2.

\subsection{Anthimis de damia (? danria).}
\paragraph{}
Amor aquæ libra 1. naptæ libra 1. sulfuris vivi libra 3. picis aridæ $\svgaaa$ 4. lac ferri libra 1. semis. Ista omnia arida tere bene, et cum liquidis, pulvere subtilissimo facto, commisce: et una hora coque, et fiet ignis; sed non secundum priorem virtutem, sed modice minus.

\subsection{De lapide olimpio.}
\paragraph{}
Lapis olimpius nascitur in petrosis locis, et est duorum colorum, niger habens guttas albas: percussus sole, sicut sulfur, ignem emittit.

\subsection{De lapide flavite.}
\paragraph{}
Lapis flavites nascitur in terra nigra: dum autem a sole percussus fuerit, infusus fiet prassinus, ex quo nascitur prassinus color.

\subsection{De lapide rubeo.}
\paragraph{}
Lapis rubeus nascitur in diversis locis; de quo et mortarium, quo aurum teritur, fiet.

\subsection{Compositio lulacis.}
\paragraph{}
Flores\footnote{Clores in MS.} caucallide, et flores elinii mundi, magma violæ duarum supra dictarum, \emph{i. e.} de viola majore partem 1. de minore partem 1. Magma autem tale, non secundum compositionem lazurin, nisi tantum cum aqua; de lilio autem veneto majore partem 1. ista magmata fiant, ambo in unum detrita utiliter, et reponantur in vase vitreo uno, magmata duo. De viola enim minore, facias semotim magma; et de lilio veneto majore, semotim facias magma. Deinde caucallide et elinii, singulorum partes 2. et de viola minore partem 1. et de majore partem 1. aluminis Egyptii spumati in libram de magma 4. specierum solidos 2. saponis ex axungia sine calce z. 1. Ista decoques modicum, et tere guati exnerviati libram 1. et commisce guatum cum coctione magmatis; et tere diligenter donec pulvis fiat, et repones ad solem, ut siccetur. Istud est lulacin leve, et lazuricunta, et boni coloris, non habens pessum, quia ex floribus est compositum.

\subsection{Compositio lazurin.}
\paragraph{}
Flores neulacis (quod Grecè ``tapsia'' dicitur, alii ``cameleonta'' vocant) colliges, et repones: deinde ungue manus tuas sapone cocto sine calce, et defrica inter manus tuas ipsos flores, et pones in vas. Post hoc, iterum unctis manibus ex ipso sapone, eosdem flores defricas diutius, et iterum repones; et hoc facies donec ipsi flores consumentur. Post consummationem autem florum collectam confectionem cooperi diligenter in vase, in loco calido, donec eam veneti coloris esse noscas. Cum autem veneti fuerit, non operies illud nisi tantum panno. Deinde sume folia viridia de uuato (\emph{sic}) exnerviato, et decoque cum urina expumata, donec solvantur predicta folia, et tamdiu coque, quoadusque consumetur urina, et pinguescat ipsa coctio; et mitte refrigerare. Deinde sume de floribus neulacis libras 3. de cocto autem guatto libras 2. de cinnabarin dimidiam z. et commiscens tere munditer, et trita cooperta in mortario stare permitte. Deinde ostreas diligenter mundatas intus et foris, et lotas a sordibus et limo, ponas in cacabo novo, et decoque usque ad pulverem; et refrigeratas tere diligenter, semotim: tollis ex ipso pulvere libram 1. et iarin mundum: mitte in aliam urinam dispumatam, terens diutius, donec turbetur, et viridescat urina, et ex eo commisce in priori mortario cum supradictis speciebus, et bene defricans repone in vase novo ad solem, una die. Postea coopertum vas oblinies diligenter, et pones in fornacem vitrarii superiorem, die 1. et exiet lazurin.

Flores neulacis, cum sapone fricatos, ut supra diximus, infunde in urina spumata, et vase cooperto pone in letamen, ut fragidet. Similiter folia guatti infunde in urina spumata; et reposita in letamen cum fragidaverit, eice de ipsis in mortario, auferens ex ipsis omnia nervia, et tolles ex ipso guatto libram 1. et de flore neulacis libras 2. et papaveris z. 2. et commixta tere, addens cinnaberin $\svgaaa$ dimidiam, et iarin $\svgaaa$ dimidiam, urinæ expumatæ $\svgaaa$ dimidiam, tere diligenter, et mittes in cacabum novum; lento igni decoque, donec consumetur et pinguescat, et fiet lazurin modicum porfirizonta.

Flores neulacis infunde aceto, et compositos in vase cooperto, ut predictum est, pone in letamen, ut fragident. Similiter guattum exnerviatum tere diligenter, et in vase novo aceto infusum operies in letamine, donec fragident ipsa folia. Postea sume de guatto libram 1. de lacca decocta in urina spumata $\svgaaa$ 1. et ex flore neulacis libram 1. de pulvere ostreæ mundæ, et lotæ, sicut supradictum est, z. 4. His omnibus, in mortario tritis, decoctione herbæ celidoniæ, quam in urinam coxeris, libram dimidiam adde, et croci $\svgaaa$ 1. Omnia trita, et in uno vase cooperta, et in letamine die 1. posita, tolles, et ad solem siccabis, et uteris.

\subsection{Lazurin aerium.}
\paragraph{}
Tolle florem de neulace, et defrica cum sapone, sicut supra docuimus, coopertum in vase, repone in letamine. Similiter facies de guatto enervato; et post dies, cum putruerint, tolle de neulace et guatto z. 1. et terens subtiliter in mortario, addens spimithin mundi $\svgaaa$ 1. et lulacin confecti et mundi $\svgaaa$ semis, cinnaberin $\svgaaa$ semis, et urinæ expumatæ cum vitriolo trito $\svgaaa$ 4. urinæ autem libras 10. et postquam residererit (\emph{sic}) vitriolum, tere ipsam urinam in mortario, quantum sufficit, et commixturam dimitte residere 2. dies. Post hoc tolles urinæ expumatæ mundæ libras 3. gallæ tritæ z. 1. commiscens dimitte infundere die 1. deinde mitte ex ipsa iotta libram 1. et, trita bene, dimitte residere ad solem, et fiet lazurin aerium.

\subsection{Item, aliud.}
\paragraph{}
Tolle Lazurin primum $\svgaaa$ 1. cinnabarin $\svgaaa$ 1. compone ut supra.

\subsection{Lazurin carnei coloris.}
\paragraph{}
Psimithin mundi triti libram 1. Lazurin z. 1. cinnabarin $\svgaaa$ 1., compone ut supra.

\subsection{Lazurin melinizonta.}
\paragraph{}
Sume neulacis flores unctos ex sapone, ut supra monstratum est, libram 1. et pone in letamen, et guatti exnervati, cum sapone, ut supra composuisti, atque in letamine macerata libram 1. Postea tritis in mortario adde cinnabarin z. 1. herbæ liciæ excoctæ, quæ coquenda est cum urina expumata, donec veniat ad 3. partem, et donec pinguescat; ex ipsa pinguedine trita libram 1. Hec omnia commixta pone ad solem, et fiet lazurin melinizonta.

\subsection{Alia lazurin.}
\paragraph{}
Cinnaberin $\svgaaa$ 1. siricum $\svgaaa$ 1. lacce coctionis $\svgaaa$ 1. lacca autem coquitur sic; Laccam subtiliter tritam coque in urina expumata bene lento igne, et ex ipsa coctione sume libram 1. lulacin solidum 1. et trita similiter, dimitte residere, et siccari ad solem.

\subsection{Item.}
\paragraph{}
Cinnabarin $\svgaaa$ 1. lulacin solidos 2. psimithin solidum 1. trita bene, sicca ad solem.

\subsection{Compositi(o) (\emph{sic in MS.}) vermiculi.}
\paragraph{}
Ex succo florum papaveris expressi $\svgaaa$ 1. cinnaberin $\svgaaa$ semis, lulacin solidum 1. hec omnia, commixta et trita, sicca ad solem.

Cinnaberin vermiculum libras 4. vermiculi terreni, qui in foliis ceri nascitur, libram 1. de coctione laccæ supradictæ libram 1. urinæ expumatæ l(ibras) 10. et sumptum utrumque vermiculum subtiliter tritum, mitte in cacabum in ipsa urina in lintheolo raro; delava vermiculum in cacabum, in quo urina decocta est, et iterum tere coctum, et delava in urina cacabi; et sic facies donec consumetur coccus totus. Deinde coques diligenter commixtionem illam, et exagitas: tunc sume ostream mundam, et bene lotam, et mittes eam in pinnatam bene coopertam, ponasque in furnum, donec refricetur, et postea tere subtiliter: atque ex ipso pulvere libras 3. mittas in coctionem predictam: bulliat bene usque in tertio. Deinde repone ad solem, ut pinguescat.

\subsection{Pandius.}
\paragraph{}
Mitte vermiculum libram 1. coccarin libram 1. (Coccarin nascitur, sicut supra dictum est, in foliis ceri) cinnaberin $\svgaaa$ 1., lazurin primi $\svgaaa$ 1. commisces; tere diligenter in mortario, et mitte de urina expumata libras 15. coques in cacabo novo, donec ad dimidiam partem veniat ipsa urina. Postea pisa grana, cum cinnaberin trita, in lintheolo delava, sicut supra continentur, donec consumetur.

\subsection{Item, pandius.}
\paragraph{}
Vermiculi libr. semis, de alio vermiculo $\svgaaa$ 6. psimithin $\svgaaa$ 6. lazurin $\svgaaa$ 6. hæc, diligenter trita, mitte in cacabum cum urina dispumata libr. 10. et mittens in lintheolo raro coccum delava in urina, et iterum delava donec expendatur coccus; et decoques donec veniat urina ad dimidiam partem, et repones ad solem.

Lulacin libra 1. cinnaberin libra 1. psimithin libra 1. ficarin $\svgaaa$. 2. hec omnia trita, et cum aqua tepida mixta, pones ad solem, donec siccentur.

Lulacin libra 1. cinnaberin principalis libra 1. lazurin libra 1. ocrea mundissima libra... quianus libra 1. hec omnia, bene trita, et cum aqua tepida mixta defricans, pones ad solem donec siccentur.

Lulacin $\svgaaa$ 3. psimithin $\svgaaa$. 9.

Lulacin $\svgaaa$ 1. ficarin $\svgaaa$ 1. quiani $\svgaaa$ 1.

Lulacin libra 1. quiani libra 1. psimithin libra 1.

Quiani libræ 2. psimithin libra dimidia; commixta tere cum urina despumata, quod sufficit; pone ad solem.

Quiani libra 1. ficarin libra 1. ocreæ libræ 2. omnia trita commisces cum urina dispumata et repones ad solem.

Quiani libra 1. de pulvere caucucecaumeni triti $\svgaaa$ 1. ficarin $\svgaaa$ 1. ocree $\svgaaa$ 1. omnia trita et commixta cum urina dispumata pones ad solem.

Cinnaberin libra 1. herbæ luciæ coctionis libra 1. croci clari mili libra 1. ficarin libræ 2. quiani libra 1. omnia trita commisces urinæ expumatæ, (\emph{sic MS.}) et sicca ad solem.

\subsection{Item.}
\paragraph{}
Cinnaberin $\svgaaa$ 6. et iotta coctionis lacce $\svgaaa$ 6. croci $\svgaaa$ 6. omnia trita et commixta repones in vase vitreo ad solem per diem, donec siccetur, et per noctem collige aput (\emph{sic}) te.

\subsection{Item.}
\paragraph{}
Cinnaberin $\svgaaa$ 1. incausti s\~{p}erii 3. $\svgaaa$ 1.\footnote{? If this should not be, incausti $\svgaaa$ 3. s\~{p}erii $\svgaaa$ 1.} teres, et commisces; repones in vase vitreo, pones ad solem, et de nocte colliges; et ita facies, donec siccetur.

\subsection{Item.}
\paragraph{}
Cinnaberin $\svgaaa$ 2. psimithin $\svgaaa$ 1. ista teres bene in mortario, et commisces cum urina expumata, et teres utiliter, vitreoque in vase repones, et cooperies in letamine per dies multos.

\subsection{Item.}
\paragraph{}
Terræ viridis $\svgaaa$ 2. cinnaberin $\svgaaa$ 1. trita commisces, ac repones quemadmodum et primum commixta cum urina expumata.

\subsection{Item.}
\paragraph{}
Terræ viridis libra 1. cinnaberin $\svgaaa$ 1. psimithin solidos 2. Ista teres in mortario cum urina expumata, et repones in vase vitreo, et pones ad solem, ut prius.

\subsection{Item.}
\paragraph{}
Terræ viridis libra 1. ocreæ $\svgaaa$ 1. cinnaberin $\svgaaa$ 1. trita, et commixta omnia cum urina expumata, mittes in vase testeo, et operies in letamine per dies 20.

\subsection{Pandius ocrei coloris.}
\paragraph{}
Ocreæ mundæ libra 1. cinnaberin $\svgaaa$ 1. ficarin solidos 3. omnia trita in mortario commisces cum urina expumata, et repones in vase vitreo, et repones ad solem, donec siccetur.

\subsection{Item.}
\paragraph{}
Lulacin, quianus, cinnaberin, lacca, equis ponderibus, trita et commixta pones in vase vitreo, et repones ad solem, donec siccetur.

\subsection{Item.}
\paragraph{}
Iotta, de coctione conchilii, libra 1. Sirici mundi $\svgaaa$ 1. omnia trita, et cum modica urina commixta, mitte in vas vitreum, et sicca ad solem.

\subsection{Item.}
\paragraph{}
Iotta conchilii, iotta de lacca, ana $\svgaaa$ 1. --- Teres primum $\svgaaa$ 1. cinnaberin, et post hoc commisces iottam conchilii, et iottam laccæ, et repones in vase vitreo ad solem, donec siccetur.

\subsection{Item.}
\paragraph{}
Iottam conchilii $\svgaaa$ 1. cinnaberin $\svgaaa$ 1. croci $\svgaaa$ 1. iottam herbæ luciæ $\svgaaa$ 4. omnia decocta in urina commixta praso sextarii.

\subsection{Item.}
\paragraph{}
Cinnaberin $\svgaaa$ 1. iotta conchilii $\svgaaa$ 1. coctio rubiæ $\svgaaa$ 1. coctio finisci simul; teres primum cinnaberin semotim; post hoc commisces omnia, et repones in vase vitreo, sicut et cetera.

\subsection{Item.}
\paragraph{}
Tolles iottam rubiæ, et addis gallæ $\svgaaa$ 3. teresque utiliter; tolles ex iotta rubiæ libram 2. et mittes in vase vitreo cum ipsa galla trita, et dimittes per 2. dies infundi: post hæc colas, et addas calcitarin z. 1. cinnabarin solidos 2. utrumque teres, et mittes ea cum supradictis rebus, et decoques donec veniat ad tertiam partem.

\subsection{Compositio viridi incausti.}
\paragraph{}
Accipe grana matura arboris caprifolii (hic est, Anglicè, ``gatetriu,'') et in mortario bene contere; post in vino diligenter fac ebulliri, ferrum eruginatum decoctioni simul adiciens. Hoc est viride et fulgens incaustum. Quod si vis pannum, vel corium viridem habere, pincello hinc desuper illine. Si vero vis ut nigrum sit, adde huic compositioni solito atramentum. Quod si vis istud, vel aliud aliquid, incaustum facere, ne decurrat, gummam cini vel prini in decoctionem pone, et simul coque.

\subsection{Ad temperandum de ivired (\emph{sic in MS. pro} ``viride.'').}
\paragraph{}
Accipe herbam, que dicitur ``greningpert,'' et ebulli bene cum cervisia aut vino, adeo ut cervis(i)a crocea sit de herba; et postea cola; deinde pulverem de viridi Greco mole cum ipsa cervis(i)a, et tantum pone de cervis(i)a, ut satis sit. Postea stet in baccili, vel cupero vase, contra solem ad maturandum.

Laminam cupri habeto, et subtilem pulverem de smeril, et, cum aliquem lapidem secare volueris, serram tuam parum humectabis saliva tua in medio, et pulverem supponas, tenendo fortiter, vel juxta, et serram adhibebis serrando lapidem.

Hoc modo polies. Pone pulverem de smeril super tabulam plumbeam, et parum de saliva humecta, et paulatim tritando, sepe respice vel ne consumatur. Secatum, hoc modo colorabis. Bisum cauculum tamdiu crema, donec in pulverem redigatur; sicca ad diem, inde fac pulverem subtilissimum, vel de levi pumice, vel de cristallo cremato fac subtilem pulverem; et extende corduan super tabulam, et cum pulvere et sputo tuo super corduan fricando lapidem colorabis. Si pumicem non habes, de antiquis lateribus, unde scutellæ fiebant, fac pulverem, qui tantundem valet; vel cum cupero filo, sine detrimento, colorabis.

\subsection{Quianus autem nascitur sic.}
\paragraph{}
Quianon uualtalasion nascitur enim in locis humidis. Nascitur enim ex rore, estivo tempore; colligitur autem sic. Tollens colliges eam, et repones ad solem, donec siccetur: postquam desiccabitur, teres bene; deinde tolles cocleam marinam majorem, et lavas bene, et terens mittes ex ipsa coclea libras 30. de uualtalasion libras 2. sapone libras 10. lazurin $\svgaab$. 3. de sapone autem ex oleo leviter cocto partem. Omnia commixta et trita repone in vase novo testeo, et operi de letamine, et dimitte diebus 60.

Perpensum ante commixtionem specierum marinarum tritum bene commisces, secundum mensuram coctionis. Quiani libra 1. psimithin $\svgaaa$ 1. ista trita et commixta cum urina expumata. Quiani libra 1. cinnabarin $\svgaaa$ 1. Hæc trita misce cum urina expumata. Hæc omnia exposuimus ex terrenis maritimis floribus, vel etiam herbis; ita exposuimus virtutes vel operationes earum in parietibus, lignis, lintheolis, vel etiam pellibus, et omnibus pictorum instrumentis, ista memoramus omnium operationes, qui in parietibus simplicem, in ligno cere commixtum suscepit lignum simplicem cum unctione collon commixtum. In pannum vero cere commixtis coloribus: in pellibus unctio collon commixtum.

Primum metallum, ex quo fit aurum, terra rufa est, amoydis subrubicunda prope ad juxta stantem illi terram est hæc, et alia similis, et dum incenditur perdit colorem, et non est arenosa, sicut illa prior. Nascitur in solanis lucis ea terra, et tale est metallum auri.

Metallum vero argenti viride est.

Metallum autem eramenti petra est viridis, quæ, dum percutis cum pirepolo, ignem emittit.

Auricalci autem petra est melina, et eodem modo ignem emittit. Metalli lapis est colore ipso gatizon.

Plumbum autem est terra fusca; lapis autem qui in ea invenitur viridis est.

Harena est, undu vitrum metallizantur; est enim lapis vitrei coloris.

Vitriolum, unde fiet terre ogrizos sunt cum cretæ, ubi verno tempore guttam colligunt ipsam, et decoquetur, ex ipsa terra fiet calcitarin; quæ autem arida, vitriolum.

Aluminis autem metallum est terra florens.

Eitarin terra est alba, facilis ad pisandum.

Sulfur ex terra nascitur, et ipse incenditur locus: coctum autem ex terra sulpherea oleo mixta coquitur.

Nitrum est sal, qui nascitur in terra, fiet in laminas in tempore cavatur.

Sal scistis nascitur similiter.

Affronitrum, verò, nascitur in loco nitri, priusquam gelet: componitur autem et aliud ex nitro; principale autem spuma, alba, ut nix. Compositum vero prius fuscum est, habet tamen eandem virtutem.

Terra sulphuritana in eodem loco, ubi sulphur, nascitur.

Argentum vivum nascitur ex terra. Nascitur et aliud ex metallo argenti in conflatione.

Auripigmentum metallum est terre; gleba est naturalis quæ in Cipro insula invenitur in metallicis, colore sub auroso, intus habet venas descissas, ut alumen scissum, et in modum stellarum fulgentes.

Prassinus terra est metallizans.

Lulax componitur ex terra et herbis.

Lazurin compositum est.

Quianus compositus est.

Ficarin compositum est.

Iarin eraminis est flos.

Psimithin plumbi est flos.

Ocrea terra est pandia, omnia colores, omnia compones.

Caucecucaumenum ex eramine fit.

Cinnabarin ex argento vivo fit.

Siricum fit ex psimithu; fit et ex plumbo.

\subsection{Composita herbarum, terræ, et lignorum.}
\paragraph{}
Chriscollon arbor est, non alta, melinum habens interiorem lignum. Nucis cortices et frutices; Cilicinæ cortices; Meliæ cortices; Ulmi cortices; Celsæ cortices.

Hec omnia tin(c)tiones sunt. Rubina silvatica, luza est. Monoclosus, galla glandis est. Drantalasis, dissobo gauzo arena est.

Resinæ omnis species ex pino et sapino coquitur. Pice recocta pecola semel: Cedria ex ligno coquitur cedrino: Mastice ex lentisco nascitur. Zigea ex zigeo. Gumma ex atrinia. Secunda gumma, ex amigdala. Lineleon ex semine lini, sicut oleum ex oliva, fit. Oleum lenticinum ex lentisco. Collium ex mari. Conchilium ex mari. Sal ex mari fit.

\subsection{(\emph{De auri pondere.}).}
\paragraph{}
Omne aurum purum cujuslibet ponderis omni argento similiter puro ejusdem tamen ponderis densius est parte sui 24. et insuper 240. quod ita probari potest. Si purissimi auri libra cum eque puri argenti simili poudere sub aqua conferatur 11. denariis, id est 24. et 240. sui parte, aurum gravius argento, vel argentum levius auro, invenietur. Quapropter, si opus aliquod inveneris formatum, cui argentum per commixtionem inesse videatur, scireque volueris quantum auri, quantumve in eo argenti, contineatur, sume argentum sive aurum, et examinato inspectione pondere non minus pensantem massam de utrovis metallo fabricato, atque utrumque opus, scilicet, et massam stantem lancibus imponito, aquisque inmergito. Si argentea fuerit, allevato opere, aurum inclinabitur: hoc tamen ita fiet, ut quot partibus inclinatur aurum, totidem partibus sullevetur argentum; quia, quicquid in ipso opere fuerit, sub aqua preter solitum ponderis ad aurum, propter densitatem, pertinet; quicquid autem levitatis ad argentum, propter raritatem, conferendum est. Et ut hoc facilius possit adverti, considerare debes tam in gravitate auri, quam in levitate argenti, denarios 11. signi(fi)care libram, sicut prima lectionis hujus fronte prefixum est.

\subsection{Compositio nigelli ad aurum.}
\paragraph{}
Sume duas partes almenbuz, \emph{i. e.} argenti, et terciam eris, et aliud tantum alquibriz, \emph{i. e.} sulfur, et aliquid majus, et mitte in fornace in caniculum, ut assetur, et tunc paulatim misce supradicto alquibriz: cum fuerit bene assatum, et misculatum, extrahe foras, et mitte in rigellum, aut in quovis loco, et stari calidum percute ut extenuetur, et dimitte frigesci. Postea super incudem cum tudone (\emph{id est}, martello) diligenter frange minutissime, ut pulvis fiat; et mitte in coculam: postea detempera atincar, \emph{i. e.} burrago, cum aqua; et cum hoc distempera nigellum, et mitte ubi vis; et desuper natroni pulverem asperge, et mitte super carbones usque quo bene currat; et ubi non vis ut currat, mitte cretam distemperatam valde subtilem. Ita factum, extrahe de fornace ut frigescat, et cum lipsatorio acerino lipsa, sed sepe aliquantulum super carbones calefac, et ita usque quo bene se habeat. Postea rade nigellum usque ad almenbuz, et iterum lipsa, ita ut melius scis, et dimitte.

\subsection{Item, nigellum ad almenbuz.}
\paragraph{}
Sume almenbuz, et aliud tantundem eris, et tantum alquibriz, quantum pensant inter almenbuz et es; et fac, ut superius dictum est de auro.

\subsection{Item, ut superius, ut colorem habeat deauratura.}
\paragraph{}
Sume urinam, et aceti modicum, et allium bene tritum, et alquibriz, ut estimas, et similiter misce in concham eream; et ibi mitte 2. laminulas, unam de cupro, et aliam de arrazgaz (\emph{id est}, plumbo) et, cum bene ferbuerit, extrahe, et intingue in aquam frigidam, ita ut fundum non tangat; sed sepe mitte, et trahe, usque quo bene coloret.

\subsection{Ut dulce fiat aurum, ita fac.}
\paragraph{}
Mitte aurum in caniculo in fornace, et misce cum eo alquibriz, et tincar, et simul confla; et fac rigellum, et mitte in salem, postea in aquam, et opera.

\subsection{Item, si vis aurum ponere in pellem.}
\paragraph{}
Si vis aurum ponere in pellem, mitte antea claram de ovo 2. aut 3. vices, in almenbuz 4. in alcazir (\emph{id est}, stannum) 8.

\subsection{Si vis colorare almenbuz.}
\paragraph{}
Si vis colorare almenbuz, sume acetum, et salem, et simul misce, et ibi infunde almenbuz calidum; postea accipe carbones tritos, et lipsa cum panno, aut cum setis.

\subsection{Si vis nectere eramen, aut auricalcum.}
\paragraph{}
Sume duas partes eris, et tertiam stagni: confla simul in fornace, et bene misce; extrahe inde ut frigescat, et inde fac pulverem subtilissimum super ferrum, aut super petram duram, ipsum pulverem misce cum oleo, non nimis rarum nec spissum. Ex hoc unge juncturam eris, aut auricalci, et super asperge pulverem natroni (\emph{id est}, alatroni), et mitte in ignem, ut calescat, et cum fuscello frica, ut bene conectet.

\subsection{Conexio auricalci.}
\paragraph{}
Natroni denarium 1. penso, Cream vini assam quantum estimas, boras denarium 1. penso, cum aqua confice, et inde line auricalcum: postea supersparge pulverem stanni assi: postea subtus carbones in fornace calcfac sicut aurum, usque quo bene conectet.

\subsection{De stagno conjunctionem. (\emph{sic.})}
\paragraph{}
Sapone partem 1. resinæ pini partem 1. natroni partem 1. borax aliquid. Ex his unge stannum, et leviter calefac, sicut scis, usque quo conectet, et calidum in aqua merge.

\subsection{Deauratio facilis.}
\paragraph{}
Accipies laminas stagneas, contingis aceto et alumine, et conglutinabis glutine cartineo; deinde sumis crocum et gluten purum (\emph{id est}, perspicuum et limpidum) infundis aquam cum aceto, et limaturis igne levi coque: cum effluxerit gluten, inunge stagneas laminas, et apparebunt tibi aureæ.

\subsection{Ad gluten stanni.}
\paragraph{}
Duas partes axungiæ, et terciam resinæ, et limaturam stanni pariter mixta, si leviter calefeceris ad ignem solidare poteris.

\subsection[206. Inauratio vase nigrum impingere, ut putes inpisatum esse.]{206. Inauratio\footnote{Inaurato, MS. Compare the title given in the List of Chapters.} vase nigrum impingere, ut putes inpisatum esse.}
\paragraph{}
Eris rubri et plumbi partes equales confla, et asperge sulfur vivum, et cum fuderis, patere ut refrigeret; mittis in mortarium, teris, adicis acetum, et facis atramentum de quo scribitur, pinguedine,\footnote{Some words appear to have been omitted here.} et scribe in auro et argento, quod velis, et cum refrigeraverit, calefacito, et erit inpistatum. Conflabitur autem ita. Carbonem sculpis, ita mitte argentum et es; confla, et cum liquescit admisce plumbum, deinde sulfur; et, cum miscueris, diffunde, et fac ut predictum est.

\subsection{Inductio exaurationis petalarum.}
\paragraph{}
Petala fiant de stanno; fiant autem sic. Solves bene ipsum stannum, et fundes paulatim in marmore, et facies petala subtilia, vel cum malleo, et pones sicut crisopetala; et decoques herbam celidoniam, ex ipsa coctione mitte uncias 3. croci libras 3. auripigmenti libram 1.

\subsection{Tinctio stagnee petale.}
\paragraph{}
Tolle croci mundi unciam 1. pigmenti optimi fissi uncias 2. mitte gummi dimidiam, et lineleon unciam dimidiam, et aquam pluvialem aut dulcem; commisce, ut bulliant simul. Commisce confectiones bene terens, tollensque cum spongia ungues ipsam petalam, et cum desiccaverit, secundo unge, et desiccatam cum onichino defrica, ut splendeat.

\subsection{Aurum probatum facere.}
\paragraph{}
Eris partes 4. argenti 1. simul confla, et adice auripigmentum non ustum, sed crudum, eurem partes 4. et, cum calefeceris, sinito ut refrigeret, et mitte in patinam. Obline argilla, et assa, donec fiat cerosa; tolle, et confla, et invenies argentum. Si autem multum assaveris, fit elidrium; si partem 1. auri adjeceris, fit aurum optimum.

\subsection{(Ad solidaturam argenti.)}
\paragraph{}
Ad solidaturam de argento 2. denarios pensante de argento, et 2. de eramine, et una medalla de stanno.

\subsection{Ad solidaturam argenti non boni.}
\paragraph{}
Accipe de bono argento 3. denarios ponderantes, et 1. obolum de stanno.

\subsection{Ad bonum argentum solidandum medium oboli.}
\paragraph{}
De commixtione puri et fortissimi xknk cum 3. qbsuf tbmkt, cocta in ejus negocii vasis, fit aqua, quæ accensa flammans incombustam servat materiam.

\subsection{De planitie, seu altitudine mensurandi.}
\paragraph{}
In primis, orthogonium hoc modo compones. Tres virgulas planas et rectas facies, primam 3. unciarum, vel pedum, seu ulnarum; secundam 4. terciam 5. Illam, quæ trium mensurarum est in altum dirigas; illam, que 4. in planum colloces; illam, quæ 5. a summitate illius, quæ in altum dirigitur, usque in summitatem illius, quæ in planum collocatur, deducas. Sic angulatim illæ tres virgulæ conjunctæ orthogonium faciunt. Virgula autem directa vocatur cathetus: collocata, basis, deducta, ypotemusa. Deinde baculum accipias, cujus altitudo usque ad oculum tuum perveniat: huic orthogonium in medio basis effigias; postmodum oculum angulo opponas in quo junguntur basis et ypotemusa. Intuitumque ad illum angulum dirigas quo jungunt ypotemusa, et cathetus; et progrediendo ac regrediendo tamdiu perambules, quousque intuitus, secundum estimationem, angulum ypotemusæ et catheta jungat summitati illius rei, cujus altitudinem queris. Hoc expleto, ab eo loco in quo tunc stas, usque ad pedem rei illius, metire spatium areæ. Ex hoc spatio quartam partem subtrahe. Reliquas tres partes, superaddita baculi mensura, quem in manu tenebas, pro altitudine teneto. Hoc autem est caute videndum, ne in aliquam partem declinet ortogonium baculo superposito; et, ut declinatio deprehendi possit, a medio ypotemusæ pendiculum demittito. Hoc, si medium basis tetigerit, nullam declinationem orthogonii esse scito.

\subsection{De lapide orcho, vel orebo.}
\paragraph{}
Lapis orchas, quem vocant Alexandrini cathmia, nascitur in humidis locis; est autem facilis ad pisandum. Est (? enim) niger, ingreditur in solidatura argenti. (\emph{sic.})

\subsection{De lapide Atriathe.}
\paragraph{}
Lapis atriathis, quem vocant leocopandium, est enim terra prassinus (\emph{sic}) in qua nascitur. Crescente autem terra, et reflorente, florem album rotundum, quadrum, acutum, ginnit. Post hoc stringit eum, et fiet lapis; florem constringit terra ipsa prassina, et fient petræ. Aliæ aurei coloris, melini, aliæ pandii, aliæ candidi. Quæ, dum percussæ fuerint, emittunt ignem, et ex ipsis egreditur argentum vivum. Aprili mense, et Maio, excalescente terra, abundantes flores cavas humidum locum usque ad geniculum, et discooperies terram, et invenies flores veteres duratos, et adherentes terræ, factos lapides. Alii enim floruerant et induraverant, et terræ non adheserunt, sed remanserunt ut margaritæ, eo quod non conjunxerunt tempus. Alii floruerunt competenti tempore, sicut nix alba, quas (\emph{sic}), cum inveneris, ita leva cum vana terra et floribus, et mittis in pila marmorea, et cum impleveris, mittis aquam, et misce bene, et terram, quæ in ea est, jacta foras, et remanet argentum vivum. Exiet et de metallo argenti, quando inchoat accendi, percurrit, et colligunt illud artifices.

\subsection{De lapide fumice.}
\paragraph{}
Lapis fumice nascitur in universis locis: tritus ingreditur in caccabum novum, et in fornacem figuli mittitur, et coquitur bene, et cooperitur diligenter, ut non ingrediatur aliqua immundicia. Post hoc eicitur, et teritur, et in compositione auri pro gemma ingreditur, in temperatione de calaina.

\subsection{Compositio auripigmenti.}
\paragraph{}
Auripigmenti triti mundi partem 1. argenti vivi z. 1. auri tremissem 1. Aurum battis, et facies petalam, et mittis ipsam petalam et argentum vivum in trullam ferream, et incendes, donec solvatur aurum, et misceatur cum argento vivo: et postea mittes auripigmentum in ipsam trullam modicum, et commixtionem argenti vivi, et coques bene, et exagita, donec fiat pandius.

\subsection{Gluten auri ad fistulas.}
\paragraph{}
De crisoclabo astici z. 1. caucucecaumenum z. 1. affronitri \roundz. saponem ex oleo sine calce z. 1. vitriolum solidos 2. aceti dimidiam \roundz. aquæ z. 1. comminue, et commisce semotim eramen; cetera in unum commisce, et pone modicum, ut tepescat gluten de canilis crisoclabi.

\subsection{Inauratio eraminis, argenti, et auricalci, crisophargia de petalis.}
\paragraph{}
Enencaus, sive encause, prima argenti et eramenti et auricalci. Battis aurum, et faciles (? facies) petalas subtiles ac tenues, et post mittes argentum vivum, et ipsam petalam solves, donec solvatur aurum ipsum. Si autem minuitur ipsum argentum vivum, adde plus, donec coquatur ipsum aurum. Deinde mittes in testam, et cum alia testa teres usque quo attenuetur et commisceatur aurum cum argento vivo; et rade vas, quod habes inaurare, et perunges modicum, et calefacies, et exprimis cum lintheolo mundo, ac totum exterges. Quod autem remanet mittes in ignem, et propria similiter, et unam et duas inaurationes in vas novum mittas. Si autem leviter semel unctum fiet, postea defrica illud ferro candente, et coloratur. Deinde cum micis panis defrica, donec elimpidet colorem.

Similiter fit et inauratio ferri, sed primo aluminatur. Tolles partem vitrioli, et modicum salis, et aceti acerrimi, in caliculo; et exinde ferrum linies, quod habes inaurare, et hic est prima inauratio.

Crocum verissimum tolles, et radices de ipso flore diutius tolles ovum, et aperiens primum proice quod exiet; sequens albumen suscipe in ipsum crocum, et terens leviter unge quod vis, et superpone petalam.

Tolles argentum vivum, commisce cum auro, sit inter rationem, et terens bene mitte in caliculum; et pone in prunas, donec siccetur vivum argentum, et remaneat aurum, quod mittis in mortario; cum pistillo ferreo teres bene, donec pulvis fiat. Tollas crocum, teras in unum; si uncia erit auri croci solidi sint 2. mittes in aquam, donec coquantur. Similiter mittes in compositione ipsam aquam de gummi; teres utiliter, et pones in ampullam, et suspende ad solem; et tolle de sole ubi volueris. Cum ipso calamo, cum quo scribis, scribe quod vis. Similiter argentum et eramen compones.

\subsection{Quo modo fiat sulfur coctum.}
\paragraph{}
Coques lardum oleo, et ex ipso tolle libras 2. et sulfuris tere libras 3. mittes in cacabo ipsam terram tritam, et bullies secundo, vel tertio, et fundes super laterem.

\subsection{Compositio affronitri secunda, quæ queritur ad gluten argenti vel eramenti.}
\paragraph{}
Nitrum Egiptium libram, 1. saponis de axungia, sine calce, libram 1. teres utiliter, et commisces. Deinde pones ad solem, vel in calido loco; utile est ad gluten auri: ad argentum autem propter mollitionem argenti componitur mollior, id est, duas partes de sapone, et unam de nitro.

Eraminis partes 2. plumbi partem 1. Tolles laminas eramenti, et derade bene, et suspende super acetum, et collectionem quam facit rades, et colliges.

Iarin partes 2. vitrioli mundi z. 4. aluminis Egiptii z. 2. Ipsum autem guattum semotim pisa munditer, Iarin vero, et vitriolum, et alumen in unum, et tolle saponem ex oleo, sine sale, dimidiam unciam, et commisce in ipsa tres species. Postea pisas commiscens diligenter cum sapone. Deinde tolle guattum ipsum pisatum, ut oportet. Commisce ipsum cum supradictis speciebus, et defricans diligenter, et dimittes diem unam requiescere.

\subsection{Confectio ejus hec est.}
\paragraph{}
Urinam mundam, et ipsam dispumatam, libram 1. commisce cum ipsis speciebus, et tere diutius. Si est cacabus ferreus, mitte in eum; sin autem, mittes in testeum, et decoques, donec veniat ad tertiam partem: postea tolles gipsum coctum, bene pisatum; mitte dimidiam unciam, et tolle coctionem: commisce cum ipso gipso, et defrica diutius, et mitte in vas. Pone ad solem, et, dum extinxerit, frange speciem, et pone illam siccare.

\subsection{Confectio ficarin.}
\paragraph{}
Tolle laccæ mundissimæ libram 1. et decoque cum urinæ dispumatæ libris 5. et decoque munditer nec dimitte supra modum dispumare, et tolle ossa cancri munda, et incende munditer, et teres quod sufficit. Commisce in lacca, tolle similam infusam in aqua; deliqua bene, z. 1. pinguis autem sit illa deliquatio, et pisa in unum bene (\emph{id est}, ossa cancri laccam), et cum ipsa deliquatione simile commiscens mitte in vas, et desicca ad solem, unde facias ficarin.

\subsection{De metallo vitri, et coctione.}
\paragraph{}
Vitri mundi de massa sume libras 5. limaturæ eramenti absque plumbo z. 2. et mitte in vas novum testeum sufferens ignem, et decoque in inferiore fornace vitrarii diebus 7. et post hoc eiciens confringe minutatim, et interim conflas prassinum tinguens.

\subsection{De metallo plumbi.}
\paragraph{}
Plumbi metallum terra fusca est, nascitur in omnibus locis, plus autem in calidis; et lapis qui in ea nascitur viridis est, sed non subalbidus, metallum autem grave. Probatio autem metalli hec est. Tolle illud, et mitte in ignem; quod, cum bullierit, et solutum fuerit, scintillas emittit. Herba, que in ipsa terra nascitur, semper marcescit, pre calitudine metalli. Colligitur autem sic propter estuationem solis, cavas terram usque ad cubitos 3. altitudine, debilis est ipsa terra; et dum cavatur desiccat, coquitur autem in fornace, quemadmodum et ferrum. Plus autem incenditur plumbum.

\subsection{Alia coctio plumbi, ex ipso metallo.}
\paragraph{}
Ipsum metallum non siccatur, sed continuo ut levatum fuerit, mittitur in fornacem ferri cum carbonibus et lento igni. Non succenditur ante noctem. In nocte autem succenditur usque ad diei quartam horam. Recoquitur autem, ut mundum fiat: missum in fornacem iterum, et ex carbonibus pini, aut abietis, coquitur per horas 3. et operabitur de eo quod oportet.

\subsection{Alia compositio vitri.}
\paragraph{}
Tolle ex eadem arena, et delava propter pulverem, et mitte decolorare, faciesque fornacem vitrarii, et facies 2. folles, et primam operationem vitri decoque, velut picis coctionem: postea tolles illud, prius recoque in fornace, sicut pix recoqui solet.

\subsection{Qualiter pelles tingantur.}
\paragraph{}
Tolle pellem depilatam et lotam utiliter, et ex galla mitte per unamquamque pellem libras 5. aquæ vero libras 21. et inmitte pellem, et exagita die 1. Post hoc lava bene, et sicca. Deinde tolle alumen Asianum, et mitte in calidam aquam; et, cum resederit, funde ex illo aquam; et mitte iterum tepentem aquam, et exagita, et mitte in ipsam confectionem unam, aut duas, pellium, et tolles et lavas illas semel. De vermiculo autem habeat unaquæque pellis dimidiam libram; quarum prima unctio hec est.

\subsection{Tinctio alithina.}
\paragraph{}
Mitte urinam expumatam in cacabum, et pone ad ignem; et vermiculum in mortario tritum ligans in lintheolo raro, mitte in cacabum calentem, et exagita quousque exeat quod exierit de lintheolo; et reliquum quod remanet mitte iterum in mortarium et tere; ligatumque in lintheolo, et in cacabum calentem positum, exagita, donec de vermiculo nichil remaneat in lintheolo.

Postea consue pelles in utris modum, et mitte ex ipsa iotta 1. confectione supradicta per unam quamque pellem libram 1. et dimidiam. Defrica bene, et dimitte totam noctem manere in ipsa confectione. Mane autem iterum confice quantum sufliciat, et, effusa iotta, lava pelles, et defrica. In eadem iotta de priore pelle tinguitur pellis pecorina, \emph{id est}, in ipsa medicatione, in qua pelles caprinæ tinctæ sunt.

\subsection{Pellis rubea tinctio.}
\paragraph{}
In calce jaceat pellis diebus 6. et mitte in sal diebus 7. et in ordeum. Deinde dimitte siccare, et tunc macera; post coque in vino vermiculum, et mitte iottam in folles una hora, et permitte siccari. Pende pellem in cantirio, et rade eam novacula, ex utraque parte; accipe sal cum farina, et melle; de fermento misce simul, ut jaceat ibi una nocte, aut duabus. Suspende ad solem, et macera; tingue eam cum eramine, et macera.

\subsection{Prassinæ pellis tinctio.}
\paragraph{}
Sume stercus caninum, et columbinum, et gallinatium: solve illud in iottam, et mitte in illam pelles depilatas, et conficies eas ibi per duos dies. Post hoc eice eas inde, et lotas dimitte siccari. Deinde sumes alumen Asianum, et, secundum quod supra docuimus de alithina faciendum, de istis facere memento. Deinde sume luzam bene pisatam, coque cum urina, coctam mitte refrigidare. Consue pelles in modum utrium, sicut diximus de alithina, et mitte coctionem in ipsos utres, et confrica bene, insufflans modicum, ut habeat ventum; et confice bene, donec combibatur medicamen. Post hoc refusa confectione, et lotis semel pellibus, iterum sume lulacin 4. $\svgaaa$ per singulas pelles, et urinam despumatam libras 6. et commixtum lulacin cum urina, mitte in ipsos utres, sicut prius iottam luzæ misisti; commisce bene, donec combibatur, et consumatur, ipse humor confectionis: et refundens, dimitte siccare quod superfuerit de iotta luzæ, et lulacin, in eo tingue pellem pecorinam, sicut prediximus de alithina; et erit prassina.

\subsection{Item, prassinæ pellis tinctio.}
\paragraph{}
Tolles pelles depilatas, sicut prediximus, et primum in stercore, deinde in alumine, conficies; et ejectas de stipterea consue in folles. Deinde tolle luzacin dimidiam libram, et urinæ dispumatæ libras 10. commisce, et mitte in folles, et confice bene, immisso modico vento, sicut predictum est. Hoc autem per dies 4. facies assidue; post 4. autem dies refunde ipsam coctionem in pelles pecorinas, confice eas per dies 5. et lotas siccari permitte.

\subsection{Melina pellium tinctio.}
\paragraph{}
In tinctione mellina confice pelles, et alumina, sicut prediximus, et lotas post alumen consue in folles: postea sume luza(c)in bene pisatam, coque cum urina bene despumata, et cum refrigidaverit, mitte ipsam iottam in folles; et confice, sicut prediximus, per dies 5. vel 6. et post hoc refundens tingue pecorinas pelles sicut supra docuimus; et post tinctionem lava et sicca.

\subsection{De porfirio melino.}
\paragraph{}
In porfirio confice pelles sicut supra, et mitte in alumen: deinde ablutas tingue melino, postea tempera coccum, et ipsam temperationem in ipsas pelles, quas tinxeras, mitte; et confice sicut supra docuimus.

\subsection{De prima tinctione pandii.}
\paragraph{}
In prima tinctione pandii, confice, eodem modo quo predictum est, pelles, et alumina: post aluminationem, abluta stipterea, tingue eas in vitriolo: post tinctionem, lava illas bene. Deinde compone vermiculum, sicut supra docuimus; ex ipsius coctionis iotta mitte in folles, et confice more solito; et refusa confectione, tingue pecorinas pelles, et lotas desicca.

\subsection{De secunda tinctione.}
\paragraph{}
In secunda vero pandii tinctione, confectis pellibus, sicut supra, et tinctis cum vitriolo, atque lotis, mitte ex iotta luzæ in folles; et confice per 4. dies.

\subsection{De tercia.}
\paragraph{}
In tercia tinctione pandii, confectis pellibus, ut prediximus, tolle iottam coccineam, et mittes in folles: exagita, et confice, sicut supra.

\subsection{(Item, de pandio.)}
\paragraph{}
Sume corallum tenue, boni coloris, rubeum, marinum, tritum, libras 2. lacca coquinum libras 1. et calcitarin \roundz. 2. trita omnia commisce, et coque cum urina; et, quando volueris tinguere, mitte ex ipsa iotta, que est confecta in urina dispumata, in folles; et confice per 2. dies. Post hoc lava bene, et desicca.

\subsection{Item.}
\paragraph{}
Tolle lubiam, et bene pisatam coque in cacabo cum urina, et addito modico alumine commisce, et dimitte refrigidare. Deinde colatam ipsam iottam mitte in folles confectarum pellium, et exagita bene, et confice die una, et lotas desicca. Post hoc sume ex iotta luzæ \roundz. 1. et luzacin \roundz. 1. et commisce, et unge faciem pellium.

\subsection{Tinctio prassina ossuum, cornuum, et lignorum.}
\paragraph{}
Rade primum quodlibet ex omnibus his, et mitte in alumen Asianum, et ossa diebus 12. cornu autem alumina diebus 8. lignum vero diebus 4. Deinde coque luzam bene, et, dum fervet, depone in illam quodlibet horum; et, dum refrigidaverit, tempera lulacerin, et depone in illud, et dimitte diebus 5. Postea eice, et lava.

\subsection{De veneti tinctione eorundem.}
\paragraph{}
In veneti autem tinctione alumina quodvis horum, sicut prediximus; et, facto lulacerin, mitte in illo, si os fuerit, 10. diebus, si vero cornu, diebus 8. si autem lignum fuerit, diebus quatuor.

\subsection{De melina tinctione eorundem.}
\paragraph{}
In tinctione vero melina, que tinguenda sint, sicut supra, alumina; et coque luzam cum urina dispumata; et, cum bullierit, depone in illam.

\subsection{De colore cinnabarin.}
\paragraph{}
Si autem colorem cinnabærin similem vis facere, sume sinopidem coctam partes 2. siricum partem 1. commisce, et tempera cum aqua.

\subsection{De cebellino, quomodo fiat.}
\paragraph{}
Cebellinum ita fiet. Tolle lignum cerrinum aut deirinum, et munda ramos ejus ex cortice; et faciem ejus leviter dola, et cooperi illud in loco cenoso annis 20. Postea ejectum de ceno dimitte in umbram siccare anno 1. et labora ex eo quod vis.

\subsection{De inauratione ferri.}
\paragraph{}
Si ferrum deaurare volueris, sume calcitarin, et alumen Asianum, equis ponderibus; salis autem et draganti tantam quantitatem, que compenset, tota trita, et commisce hec omnia cum aqua, vel aceto acerrimo; mittens in cacabum eneum; et bulliat una hora diei. Post extersum et pumicatum diligenter ferrum, cum hac confectione perunge, in eo loco ubi volueris deaurare; et, cum modicum in hac confectione patieris, terge ferrum, et habebit colorem eneum. Deinde lima cum lapide onichino ferrum; et, si limando perdit colorem, tunc iterum tingue; si vero inauraturam comprehendere voluerit, commisce cum medicamine equaliter, et unge sicut prius.

De abluta tauratica in qualem vis pannum facere, et sic lixandum est cum onichino lixa.

\subsection{De petalo aureo.}
\paragraph{}
Lineleon \roundz. 5. galbani \roundz. 2. terebentinæ \roundz. 1. picis Spanæ \roundz. 1. Istas 3. species, (\emph{id est}, galba- 

num, terebentinam, picem Spanam) solve in unum cum modico lineleon. Postea vero cum orientali croco \roundz. 2. libani \roundz. 3. floris populi primotici \roundz. 2. vernicis \roundz. 2. lineleon et uricella commisce, et ad mosana colas. Postquam tota simul fervent misce ibi de gumma cerasii \roundz. 1. Hec omnia simul commixta bullire facias in lineleon mensura 3. unciarum: post coctionem autem cola per lintheum, et misce species supradictas (\emph{id est}, galbanum, terebentinum, et picem Spanam) et, si quid vitium postea habuerit, quo siccari non possit, addito masticæ quantum volueris, aut \roundz. scilicet, aut dimidiam, emendabitur.

\subsection{Lucida quomodo fiant super colores.}
\paragraph{}
Lineleon \roundz. 3. terebentinæ \roundz. 3. galbani $\svgaab$. 2. liricæ \roundz. 2. libani \roundz. 3. mirræ \roundz. 3. masticæ \roundz. 3. vernicis \roundz. 1. gummæ cerasii \roundz. 2. flores populi \roundz. 2. gummæ amigdalæ \roundz. 3. resina sapini \roundz. 2. Hec omnia pisanda sunt, et crib(r)anda; et cum supradicto lineleon in gabbata auricalci, mittendaque in furnum calidum, ubi sine flamma coquentur, eo modo ut foras non exeant. Cocta autem colanda sunt cum lintheo mundo; et, si rara venerint, decoque ea usque dum spissa fiant. Postea vero quelibet opera, vel picta, vel sculpta, volueris, illucidare poteris, lucidata autem siccare.

\subsection{De crisografia.}
\paragraph{}
Aurum obrizum lima tenui limâ, et in mortarium porfireticum mitte acetum acerrimum, et teres pariter, et lavas; quamdiu nigrum fuerit, effundes, tunc demum mittes aut salis granum, aut nitrum, et sic solvitur. Postea scribe, et litteras polies. Sic omnia metalla solvuntur.

\subsection{Aureis litteris scribere.}
\paragraph{}
Sume laminas aureas, et argenteas, et solve in mortario, terendo cum sale Greco vel nitro, donec non compareat. Deinde mittes aquam, et infundes, et iterum mittes sal, et ablues: et, ubi purum aurum remanet, adicies eris floris modicum, et fel taurinum, et tere simul, et scribe: litteras poli. Si vero vis ut diffusum sit, et abundantius scribere volueris, tere separatim auripigmenti scissilis 4. partes, elidrii partem 1. et, cum cribraveris illud, ex eo tantum quantum equale sit auro contere pariter; et scribe, et, cum siccaverit, litteras poli. Ex hoc autem et in vitro, et in marmore pingere poteris, eodem modo quo scribis cum auro.

\subsection{De inauratione pellis.}
\paragraph{}
Tolle pellem rubram, et pumica eam diligenter; deinde lava eam aqua tepida, quoadusque limpida aqua egrediatur: postea tensam in cantario limniza usque 4. vices: deinde tende eam in axem que habeat superficiem mundam, et cum ligno mundo co-equa eam diligenter: postquam autem desiccata fuerit, tolle albuginem ovi, et spongiam mundam tingue in ipsam lacrimam, et induc semel per ordinem. Si autem non sufficit, iterum ungues; et, cum siccata fuerit, pone petalum, deinde tingue spongiam in aqua, et preme petalum ad pellem; et, cum siccata fuerit, poli. Deinde cum pelle munda desuper frica, et iterum poli, et similiter dragantum inauratur, ita tamen ut mittas in aquam in noctem, quoadusque solvatur.

\subsection{Confectio maltæ.}
\paragraph{}
Olei libras 8. casei libras 8. interiora ovorum 30. albuminum z. calcis mundæ modium dimidium, lini mundi minutatim incisi libram.

\subsection{De ere albo.}
\paragraph{}
Mitte es album in fundum caluculi, et in summo vitrum pone, et sic confla illud. Conflatum autem cum fundere volueris, fuscello remove vitrum, et non perdet colorem.

\subsection{De lacca, quomodo laboratur ad pingendum ligna, seu parietem.}
\paragraph{}
Primitus tere laccam, et inde elige fustes et spurcedinem; deinde mittas in molam, et mole subtiliter: deinde accipies urinam humanam utriusque sexus, et mitte primitus in caldariam, et dimittas bullire, donec totum consumetur, usque in terciam partem; et semper spumam frequenter tolles. Posthæc mittes laccam, et bullire facies; deinde accipies alumen mundissimum, et terens misce in supradicta lacca. Tunc accipies modicum pannum, et tingue frequenter, donec color bonus appareat. Deinde aquam mittas in vascula, et labora: lapidem qui fit in aqua, proice eum, quia nichil est. In libras 5. de lacca mittes aluminis \roundz. 5. urine sextaria decem.

\subsection{De calce et arena.}
\paragraph{}
Preterea scire est necessarium construenti quæ calcis et arene natura sit utilis. Arenæ ergo fossicæ sunt tria grana,\footnote{Sic, sed ? pro ``genera.''} nigra, rufa, cana. Omnium precipua rufa, et melioris meriti. Sequentis est cana; tertium locum nigra possidet. Ex his ergo que compressa manu edit stridorem, erit utilis fabricanti. Item, si panno vel lintheo candidæ vestis impressa et excussa nichil maculæ relinquet, aut sordis, egregia est. Sed, si fasilis (\emph{sic}) arena non fuerit, de fluminibus, aut galera, aut littore, colligetur. Marina arena tardius siccatur, et ideo non continue, sed intermissis temporibus, construenda est, ne opus onerans corrumpat. Camerarum quoque tectoria falso humore dissolvit. Nam fossiles tectoriis, et cameris, et celeri siccitate utiles sunt; melioresque si statim cum effossæ sunt misceantur. Nam diutino sole, aut bruma, aut imbre, vanescunt: fluviales tectoriis magis poterunt convenire. Sed, si necesse est ut maris arena, erit commodum prius lacuna\footnote{? lacivia.} dulcis humoris immergi, ut vitium salis, aquis eluta suavibus, deponat. Calcem quoque albo saxo duro, vel Tiburtino, aut columbino, fluviali quoque remus aut rubro, aut frongia, aut postremo marmore. Quod erit ex spisso, et duro saxo, structuris convenit. Ex fistuloso vero, aut molliori lapide tectoriis adhibetur utilius. In duabus enim arenæ partibus calcis una miscenda est. In fluviali vero arena, si tertiam partem testæ contusæ cretæ addideris, operum soliditas mira prestabitur.

\subsection{De latericiis parietibus.}
\paragraph{}
Quod si latericios parietes in pretorio facere volueris, illud servare debet, ut, perfectis parietibus in summitatibus, que trabibus subjacebit structura testea cum coronis prominentibus fiat, sesquipedali altitudine, ut, si corruptæ tegulæ aut imbrices fuerint, parietem non penetret pluvia. Deinde providendum est, ut siccis et asperatis parietibus latericiis inducatur tectorium, quod humidis ac levibus adherere non poterit; et ideo tertio debebis eos prius obducere, ut tectorium sine corruptione recipiant.

\subsection{Confectio Saphiri.}
\paragraph{}
Argentum et sulfur incendatur simul; postea claro vitro museo libris 2. et de supradicto safiro \roundz. 3. simul coctum iacintinos facies lapides.

\subsection{Confectio vitri rubei.}
\paragraph{}
Sumes argillam ferri, et coques igne, et postea infundes vinum; et, venam rubeam inveniendo, contunde eam pistillo, mixto cupro; et sic infunde vitrum in fornace.

\subsection{Aliter.}
\paragraph{}
Viridi vitro, cupro mixto, superfundas contusam petram ardennanam.

\subsection{Gluten argenti et auri.}
\paragraph{}
Tolle 2. partes argenti, et tertiam cupri, ac purum staminis adiciens, argentum vel cuprum bene potes conectere.

\subsection{Aliud.}
\paragraph{}
Tolle etiam 3. partes auri, quartam cupri, fundesque illud simul.

\subsection{Aliud gluten stanni.}
\paragraph{}
Duas partes axungiæ, et tertiam resinæ, et limaturam stanni pariter mixta solidare poteris, si leviter calefacis ad ignem.

\subsection{(De) Denario auri.}
\paragraph{}
Tabula cupri, que 10. pollices habeat in latitudine, et totidem in longitudine, denario auri deaurari potest.

\subsection{De vivo argento.}
\paragraph{}
Si vivum argentum adheserit operi aureo, quod non potest in ignem mittere, accipe urinam hominis, et misce simul atramentum et salem, et fac spissam pinguedinem, et inde mitte super vivum argentum, quod adheserit (\emph{vel} adhesit) aureo operi, et dimitte aliquandiu desuper; tuncque exterge, et non apparebit. Deinde ferro defrica, ut scis.

Accipe duorum denariorum et oboli pensum de purissimo cupro, et 1. de argento, simulque funde; et post, tunde tenuissimum quantum potes, et post arde petram vini diligenter; et, cum volueris solidare, accipe inde, et distempera cum aqua, tamen spissum; tunc incide minutissime cuprum illum tenuissimum, et mitte inter juncturam, et super eam quam vis solidare; et tunc superadde illam spissam pinguedinem, quam fecisti de aqua et petra vini; et tunc in ignem pone, et suffla.

Accipe plurimum atramenti, et valde arde; et de sale bono, ita ut duæ partes sint salis, et tertia atramenti; et post, misce simul, et distempera cum optimo aceto, et tunc lava bene deauratum opus, et lini ex omni parte de supradicta pulte; et post, mitte in ignem, et calefac usque quo rubeum fiat, et post, extingue in eneo vase, et tunc de seta exterge. Si autem prima vice non defecerit, adhibe sal, et reitera.

\subsection{De sagitta plumbea, ad incendendum.}
\paragraph{}
Semel, vel secundo, vel tertio, solves plumbum, et mundas ex omni sorditie, et dimittas illud quoadusque colligat v. n. n. Post hæc adducis lenticulam, et pisas utiliter, et infunde aceto, et spumam quam emittit tolles, et intingue sagittam; et ex ipsa spuma acue sagittam in plumbo, veluti in cote, donec elimpidetur; et ex eo plumbo ipsa sagitta perungatur.

\subsection{Aliud toxicum unde sagitta in pugna toxicatur.}
\paragraph{}
Sudorem equi, quem in dextera parte inter coxas habuerit, sume, et intingue sagittam. Hoc expertum est utiliter.

\subsection{De sagitta, quæ ignem emittit.}
\paragraph{}
Sagitta ad emittendum ignem, tribulus es pertusus. Confectio autem ignis talis est. Naptæ \roundz. 1. stupi solidi 2. picis conditæ solidi 4. sulfuris vivi mundæ guttæ solidus 1.\footnote{? \emph{if} ``climatidos'' \emph{is not omitted here.}} semis, salis marini solidus 1. olei ex olivis solidus 1. visci asperi solidus 1. lapidis gagatis solidus 1. saponis ex oleo solidus 1. Ista tollens, et mittens in mortario marmoreo, cum tritorio ferreo teres utiliter. Primum quam mittas naptam, stupium, et picem, et climatidam, et sulfur vivum, et gagatem, et sal marinum tere subtiliter. Post hoc mittes oleum ex olivis, et saponem, et lac mulieris, solidum 1. et teres subtiliter omnia in unum, nucis interiora 4. tere diligenter, et compones. Lac autem pingue sit.

Hec est autem tinctio sagitte: stupam lini mollem intingue, et tolle funiculum subtilem, qualem possit capere ipsa pertusa: de reliquo perunge sagittam, sicut provideris; et, cum tetenderis arcum, incendes igni, et continuo dimittes sagittam ubi volueris incendium sagittandum.

\subsection{Alia brevis.}
\paragraph{}
Alia sagitta, que ex modica compositione ignem emittit, sulfur vivum, colofonia, equis ponderibus, oleum ex nuce; conficies et disponas, sicut prediximus.

\subsection{Alia.}
\paragraph{}
Sagitta in toxo palestra venenata, ut non incendat canale. Vesti ipsum canale ex ere. In sagitta vero tribulo utere secundum priorem compositionem, et petra focaria, que ignem emittit, pertusa, non equata, sed aspera, quo possit inherere ipsa compositio.

\subsection{De rapidissimo compositione.}
\paragraph{}
Ex sulfure libram 1. naptæ libram 1. stupii libram 1. climatida libram 1. pice condita libram 1. lactis mulieris \roundz. 1. olei porcini \roundz. 1. lapidis gagatis \roundz. 1. resinæ conditæ libram 1. sulfuris vivi libram 1. picis liquidæ \roundz. 1. semis, cedera \roundz. 1. olei ex olivis \roundz. 1. sulfuris cocti \roundz. 1. auripigmenti \roundz. 3. nitri \roundz. semis: omnia collecta arida diutius tere, et post hæc tollens omnia humida commisces: tere diligenter, et compositis utere, perungens petram, et omnes pertusas ejus replens; et, imponens in petrariam, immittas ignem, et dimittes celeriter.

\subsection{Compositio arietis ad muros.}
\paragraph{}
Anteriores pedes 3. facias cubitorum 5. medios cubitorum 4. posteriores cubitorum 3. Rotæ autem altæ unius semis palmæ; gressæ 4. z. circinas, et in medio pertundis, secas columnas, et in minutis rotis, usque ad 4. unciarum co-operiens, et super connexionem facies, et configes cum meura astringens arietes, et contexes funibus, proteges eum corio, et super filtris co-operies, et super filtra, coria; et super coria, arenam \roundz. 4. et super arenam, lanam; ut non moveatur ipsa arena, et desuper coria. Tales autem habeant ipsæ columnæ cardines ut non moveantur, quia configuntur intus; et rotis suppositoria suppones; et ipso ingenio conjungas muro, et labores indubitantur.

\subsection{Quomodo debeat zelum arietis incendere.}
\paragraph{}
Compones cacabum non coctum compositione de damie, ignem apponens, et in ipso zelo mittas, et incendas ipsa coria, et lanam. Post hoc remanet arena, et ipsa confixa, eo quod non ardeat: deinde lapidas per moles ipsam arenam; compones iterum similem cacabum ex ipsa compositione jactas ipsum zelum; et, si propter multitudinem arenæ non ardet, jacta alios lapides, et compone alium cacabum, et jacta.

\subsection{De 4. formis specierum.}
\paragraph{}
Composito 4. formis specierum plus utilis ad incendendum, naptæ, picis, stupii, climatidos.

\subsection{Compositio naptæ hæc est.}
\paragraph{}
Naptæ scl. purgamenta lini, sive amurca olivæ, naptæ mundissimæ libram 1. sulfuris vivi libras 2. milino \roundz. 4. salis marini penso, solidos 4. colofoniæ mundæ liquatæ \roundz. 2. solidum 1. peculæ solidos 4. aureis 6. picis duræ \roundz. 1. olei terebintini dracontoides \roundz. 1.

\subsection{Compositio olei terebintini hæc est.}
\paragraph{}
Olei communis partem 1. olei laurini partem 1. solidos 2. cedercæ cedrinæ solidos 2. picis cipressinæ solidos 2. masticæ solidum 1. lapidis gagatis solidum 1.

\subsection{Hec est compositio naptæ 12. specierum. Confectio autem (ta)lis est.}
\paragraph{}
Teres omnia arida, humida commiscens postea diligenter cum pulvere aridorum, et defricas omnia bene: post hoc repones in vase testeo bituminato, et sic dimitte 2. dies, aut 3. et post hoc, tollens, calefacies modicum, ut bulliat, et dimitte residere.

\subsection{Confectio picis hæc est.}
\paragraph{}
Picis aridæ \roundz. 1. denar. 1. sulfuris vivi denar. 1. resinæ denar. 1. iscira (alii dicunt florem aquæ, alii oleum aquæ, alii celidonia, Alexandri autem, amorem aquæ. Nascitur autem in aqua ubi alba terra est, terra rusea, terra nigra, exeunte autem de aqua fiet milinii super aquam in circuitu exitus aquarum). Est autem aqua gravis, et egrota pro terra quæ florem generat. Sic autem colligetur. A Martio vel Aprili mense, si est calidus locus, usque ad Octobrem, colligitur. Tolles autem mollissimam lanam lotam, et inpones super aquam, et exprimes in vase vitreo, habente pertusum modicissimum, veluti acu factum; quem, cera clausum, pones ad solem diebus 10. ac noctibus, immobilem; et post hoc, ablata cera, aperies ipsum pertusum, supposita lana munda deliquatur aqua, et remanet amor aquæ. Tunc de amore aquæ mittes \roundz. 1. balsami mundi \roundz. 1. aurei 6. oleum silicum (alii oleum ricinum dicunt, alii lancidis, alii vero viscum) solidum 1. picis cipressinæ solidum 1. aureas 12. picis pineæ solidum 1. semis, saponis ex oleo solidos 2. nitri solidum 1. selinistreo (alii rodica, alii pancii, alii rusticum, alii gumma, alii matican, alii thimon, alii tricas, alii tricoselinon; nascitur autem in aqua velut apium, et in parietibus, ubi calx est, et dicitur herba capillatia) arida trita \roundz. 1. alochias (alii sticis, alii calmidam, alii cathan, alii ageropa, alii marcianin, Alexandrini autem Scaramandia, isauri papati herba est subtilis, ramosa, spina involuta, albidiante; multi exinde accendunt candelas: nascitur autem in locis petrosis et asperis; folia ejus, sicut mirtæ, spissa) aridæ tritæ solidos 4. denar. 1. Robaticis (alii exmilax, alii telacion, alii quisnasbatu, alii cucudera: nascitur sicut rubus, fortiores habens ramos, et spissus est; fructus vero ejus est similis zizifæ, plus rotundas, pro quo vocaverunt gratiles, (?) zizifa agrestis; habet intus grana triangula pilosa:) ipsa grana siccata et trita solidos (\emph{blank in MS.}): omnia arida, trita semotim; deinde commixtis cum reliquis, adde zizea solidos 2. et, terens omnia in unum, commisce, et repone in vase testeo bituminato; et, cum bullierit, lento igni semel, fiet compositio sicut pix.

\subsection{Compositio stupii hæc est.}
\paragraph{}
Lac ferri $\svgaaa$. denar. 3. sulfuris vivi $\svgaaa$. 1. aureos 12. aluminis Persiatici denar. 3. gumma de atrinia solidum 1. amoris aquæ libram 1. balsami $\svgaaa$ 4. oleo ex olivis denar. 1. lapidis gagatis, aureas 4. semis tigea aureas 2. grana 3. cedrea de cedro pingue solidos 7. aureas 2. sulfuris melini \roundz. 3. olei laurini aureas 12. resinæ mundæ terebintinæ melinæ aureas 4. picis de pino frigidæ \roundz. 1. apallis, (alii ramitan, alii cordenan, alii daucallida, alii maragnin, Egiptii fondella, pagani tinctio, alii polligalla; herba est alta, si multum usque ad geniculum, folia simil(i)a mirti, plus majora; alia est major ticior, alia autem subtilior, ticior minor herba est, unius palmi altitudinis; majorem autem, ubicunque rumperis, lac exiet, rami ejus rotundi, folia spissa,) lactis ejus collecti et sicci aureas 4. semisses, brachia, (alii tutumallum, alii da...\footnote{Cut off by the binder.} alii leptugalia, alii polligala, alii leptotui; similis est enim prioris, et non sic alta, prior enim in principio ramos eicit, ista vero sursum ramos habet rotundos, folia subtilia, et plus rotunda quam prior, propter quod vocaverunt eam Ethiopes surganam;) ex ipsa collecta, et desiccata, et trita, \roundz. 1. aurea 1. semis; oia (alii laucia, alii sehum marinum, alii briania, alii spumam marinam, alii eleoboron, alii mag(un)tiani, Egiptii drautia; nascitur in omnibus locis, plus autem in durissimis, fungus est rotundus, pagani vocant amanita, desiccatum ubicunque percusseris pulverem levat mulmum, ideo vocaverunt eam grirovagam, rotundus nascitur totus in terra; qui ut siccus conculcatur, aut percutitur, exiet pulvis, et reliquum durum corium remanet, quasi ovum decoctum. Ex ipsis fungis, cum corio et pulvere, 7. elleborum nigrum siccatum, et tritum, solidum 1. gumma de arbore elanton (quod (\emph{sic}) est abies) $\svgaaa$ 1. Omnia ista trita, arida, et in pulverem redacta commisce diligenter cum humidis, et repone in vase fictili bituminato, et bulliat lento igni.

\subsection{Compositio climatidos hæc est.}
\paragraph{}
Sarmenti vitis agrestis florem desiccatum et tritum solidos 4. aureas 3. lac ferri \roundz. 1. naptæ \roundz. 1. sulfuris vivi \roundz. 1. semis, resinæ solidos 2. pisas olimpias solidos 2. omnia simul commiscens repones in vase testeo bituminato, et semel bulliat lento igni, sopitum est.

Commixtio 4. specierum supradictarum (\emph{id est}, naptæ, stupii, picis, climatidos) plus utilis ad in(cen)dendum.

Naptæ bonæ et humidæ libras 2. climatidos libr. 2. Commixta omnia in cacabo eneo, et decocta lento igni, inples pinnatam crudam de catia erea; hoc argentum plus est utile ad incendendum.

\subsection{Remedium ad extinguendum.}
\paragraph{}
Si arserit ignis, necesse est ex arena et sulfure extinguatur; si plus arserit, arenam, urina infusam, immittes.

\subsection{Quomodo fiat sapo ex oleo, vel sepo.}
\paragraph{}
Crati baticie de minutulis virgulis, sive spisso et forti colatorio, supersterne bene arsum cinerem de bonis lignis; et superfunde leviter aquam calefactam, ut guttatim transeat; et lexivam subtus mundo in vase recipe, et secundo vel tertio per eundem cinerem cola, ut fortis lexiva fiat, et colorata; et hæc est prima lexiva saponarii, quam, cum bene depuraverit, mitte coquere; et, cum diu bullierit et spissari ceperit, addito oleo sufficienti, move optime. Quod si cum calce facere volueris, mitte ibi modicum calcis bonæ; et si sine calce esse volueris, sola predicta bullire permitte, donec excocta sit lexiva, et in spissitudine redacta, et post in loco apto refrigerare permitte quicquid ibi lexivæ, vel aquosum, remansit: que depuratio secunda lexiva saponarii dicitur. Postea, per 2. vel 3. vel 4. dies, spatula exagita, ut bene cohereat et exaquetur, repone usui. Si vero de sepo facere volueris, eadem erit actio, sed, loco olei, mittes sepum pecorinum bene contusum; et adicies de simila, ad estimationem, et coquentur ad spissitudinem, ut predictum est. In secunda vero lexiva, quam dixi, mittes sal, et coques donec exsiccetur, et hoc erit affronitrum ad solidaturam.

\subsection{Color albus.}
\paragraph{}
Stannum libras 10. plumbum libram 1. in pulverem redactam, alumen Asianum libras 10. arena citrina libras 8. et semis; et fac fornacem, et da ei ignem; et post coctionem frange, et cribella istam cocturam; et postea adjunge arenam similiter cribellatam libras 9. et semis, et postea plumbum libras 6. et semis, et 5. libras de stanno similiter, sicut scripsimus.

\subsection{De amido.}
\paragraph{}
Amidum medulla est de frumento media libræ mixta in aqua calida 5. uncias, et mediam de vitreo safiro, et aquam quantum sufficit.

\subsection{De colore veneti.}
\paragraph{}
Color venetus. Alumen scissum $\svgaaa$. 10. et arene $\svgaaa$. 5. et lapides albos et rotundos 4. uncias, arsa et cribellata, et 4. uncias de plumbo, et unam $\svgaaa$. et mediam, et argenteos 15.

\subsection{De lapide Egrippo.}
\paragraph{}
Si inveneris lapidem de terra de Egrippo, inmitte 4. libras de plumbo, arsum et cribellatum; et de terra purpurea libram 1. mixtam in ipsa confectione, et de petra de Corintho mitte libras 2. et de plumbo 1. et de terra purpurea libram 1.

\subsection{Compositio sisami.}
\paragraph{}
Sisami compositio. Mel album et purum, in stannato ad modicum ignem appositum, incessanter spatula exagita, intercisis vicibus ad ignem, et ab igne, depositum; et spatiosius exagitatum, iterum atque iterum igni appone et depone, sine intermissione exagitans, quoad spissum fiat, et conglutinosum. Cumque satis spissatum fuerit, super marmor effusum paulisper refrigerare permitte; post, ad clavum ferreum suspensum, et crebro et minutule extensum, et replicatum, donec albescat, ut oportet. Tunc retortum et formatum super marmor colloca, et usui ministra.

\subsection{De zuchara.}
\paragraph{}
Simili autem actione, et coctione, de zuchara in stannato, pauca aqua infusa, et, cum bullierit, dispumata, et bene colata colatorio; et sic, adhibitis quibus scis speciebus, incessanti agitatione ad spissitudinem ducta, in subuncto modico oleo marmore expanse diffundes, et refrigeratis marmoricis caute, ad manus sejungens a marmore, usui reservabis.

\subsection{De penidiade.}
\paragraph{}
Penidias vero modo sisami post dispumationem et colaturam zucharæ, sed sine exagitatione, percoctas ad clavum, ut dictum est, malaxando, conformabis, concidendo forficibus.

\subsection{Azur quomodo molatur.}
\paragraph{}
Azur mole cum sapone; post lava bene aqua.
\begin{quotation}
Hinc sextam placuit fingi, siliquamque vocari;

Ultimus est calcus, ciceris duo granula pensans.
\end{quotation}
\begin{figure}[H]
\centering
\includegraphics[width=0.95\textwidth,keepaspectratio]{005-trans.png}
\caption*{ag.\hspace{3mm} berch.\hspace{3mm} cen.\hspace{3mm} derhu.\hspace{3mm} eg.\hspace{3mm} feu.\hspace{3mm} genue.\hspace{3mm} he.\hspace{6mm} cer.\hspace{3mm}  la...\footnote{The margin of the M. is cut.}}
\end{figure}
\begin{figure}[H]
\centering
\includegraphics[width=0.95\textwidth,keepaspectratio]{006-trans.png}
\caption*{man.\hspace{4mm} net.\hspace{4mm} os.\hspace{4mm} perd.\hspace{4mm} cui.\hspace{5mm} rat.\hspace{5mm} sigil.\hspace{3mm} tir.\hspace{3mm} ur.\hspace{3mm} xen.\hspace{3mm} uir.}
\end{figure}
\begin{figure}[H]
\centering
\includegraphics[width=0.15\textwidth,keepaspectratio]{007-trans.png}
\caption*{super su(nt)}
\end{figure}
\clearpage
\begin{table}[H]
    \centering
    \small
    \begin{tabular}{l l}
        $\svgaac$  &  Assis 12.                       \\
         $\svgaad$  &  Deunx 11.                       \\
         $\svgaae$  &  Decunx 10.                      \\
         $\svgaaf$  &  Dodrans 9.                      \\
         $\svgaag$  &  Bisse 8.                        \\
         $\svgaah$  &  Septunx.                        \\
         $\svgaai$  &  Semis.                          \\
         $\svgaaj$  &  Quincunx.                       \\
         $\svgaak$  &  Triens.                         \\
         $\svgaal$  &  Quadrans.                       \\
         $\svgaam$  &  Sextans.                        \\
         $\svgaan$  &  Uncia.                          \\
         $\svgaao$  &  Dimidius obolus.                \\
         $\svgaap$  &  Plenus obolus.                  \\
         $\svgaaq$  &  Duo oboli.                      \\
         $\svgaar$  &  Tres oboli.                     \\
         $\svgaas$  &  Quatuor oboli.                  \\
         $\svgaat$  &  Quinque oboli.                  \\
         ~  &  Oboli 6. significant denarium.  \\
         $\svgaau$  &  Siliquæ 8. H.                   \\
         ~ &  Solidum unum. H.                \\
         ~  &  Dragma 1. uncia.                \\
         ~  &  Medius solidus. B. \\
    \end{tabular}
\end{table}
\paragraph{}
Duæ partes cineris quercini, cum tertia calcis querci, cum glutinentur bene, agitate cum aqua frigida: post, totum pones in cophino fortiter impressum, faciens desuper aquam locum ne discurrat, quo pones aquam frigidam, bis et ter, secundum consumptionem suppositi cineris, et calcis. Que non cito, sed sequenti die manabit, cui suppones folia lauri, vel alia congrua, ut postea fluant in aliud vas, et hoc est capitellum. Si vero saponem perficere volueris de priori aqua secundam pones; qua decursa, pones et tertiam, que bona erit quousque alba fiat; post, liquefacies sevum, colabis, et colatum, et superius si opus est, purgatum, cum ultima bullies; quo spissante secundum pones, et eodem modo de prima, aut si per unum diem oculos populi tritos infuderis, et post, expressos ejiceris, sapo rubeus et melior erit; et hic est Gallicus sapo, et spaterenta, \emph{id est}, acutus.

\begin{table}[H]
    \centering
    \small
    \begin{tabular}{|r|r|r|r|r|}
    \hline
        1 & 2 & 3 & 4 & 5 \\ \hline
        2 & 4 & 6 & 8 & 10 \\ \hline
        3 & 6 & 9 & 12 & 15 \\ \hline
        4 & 8 & 12 & 16 & 20 \\ \hline
        5 & 10 & 15 & 20 & 25 \\ \hline
        6 & 12 & 18 & 24 & 30 \\ \hline
        7 & 14 & 21 & 28 & 35 \\ \hline
        8 & 16 & 24 & 32 & 40 \\ \hline
        9 & 18 & 27 & 36 & 45 \\ \hline
        10 & 20 & 30 & 40 & 50 \\ \hline
    \end{tabular}
\end{table}
\begin{center}
(\emph{A written leaf has been torn out here.})
\end{center}
\paragraph{}
Stagni $\svgaaa$. 9. c(u)pri $\svgaaa$. ar(genti) 6. $\svgaaa$. simul funde, per figuram arragab. ad libitum tuum manabit, et stabit fons. Per eandem, ciphus potum aut reddet aut retinet.

Ex figura .a. si prius potaverint boves, sufficiet et bobus et equis. Si prius equi, deficiet et bobus et equis.

Ex eadem a dolio in alveum exibit vinum, donec impleat alveum: impleto alveo, nil exibit a dolio.

Idem in lucerna et oleo per arenam et clavum et aquam a domuncula exibit fantasma et redibit. Per ignem et aquam subclusis ventilabit coridon.

Per figuram .a. ablata lancea, exibunt milites a castro, et intrabunt stridente lancea.

Ovum in calce, calcem in puteo. Quatuor circulis imis inter alios, exposita diametrorum formatione, volventibus, vaseque interiore suspenso, quocunque modo volvantur, nil effundetur.
\clearpage
\begin{figure}[H]
\centering
\includegraphics[width=0.95\textwidth,keepaspectratio]{004-trans.png}
\end{figure}
{alpha\hspace{11mm} beta \emph{vel} uita\hspace{2mm} gamma\hspace{5mm} delta\hspace{5mm} ebrachi\hspace{8mm} zita\hspace{10mm} ita}
\begin{figure}[H]
\centering
\includegraphics[width=0.95\textwidth,keepaspectratio]{027-trans.png}
\caption*{a\hspace{24mm} b\hspace{14mm} g\hspace{14mm} d\hspace{14mm} e\hspace{14mm} z\hspace{10mm} i longum}
\end{figure}
{\hspace*{47mm}(t)hita.\footnote{Margin of the MS. cut.}\hspace{5mm}  iota.}
\begin{figure}[H]
\centering
\includegraphics[width=0.25\textwidth,keepaspectratio]{028-trans.png}
\caption*{th\hspace{10mm}  i}
\end{figure}
{cappa\hspace{8mm} lappa\hspace{8mm} mi\hspace{11mm} ni\hspace{12mm} xi\hspace{8mm} brachi\hspace{8mm} pi\hspace{11mm} ro}
\begin{figure}[H]
\centering
\includegraphics[width=0.95\textwidth,keepaspectratio]{029-trans.png}
\caption*{}
\end{figure}
{\hspace*{49mm}(sig)ma\footnote{Margin of the MS. cut.} tau}
\begin{figure}[H]
\centering
\includegraphics[width=0.25\textwidth,keepaspectratio]{030-trans.png}
\end{figure}
{uui\hspace{18mm} phi\hspace{18mm} chi\hspace{18mm} psi\hspace{16mm} oto\hspace{8mm} maga}
\begin{figure}[H]
\centering
\includegraphics[width=0.95\textwidth,keepaspectratio]{031-trans.png}
\caption*{\hspace*{26mm}f\hspace{18mm} ch\hspace{18mm} ps\hspace{18mm} o\hspace{10mm} magnum}
\end{figure}
\subsection{Ad vitrum incidendum.}
\paragraph{}
Cum acri urtica ubera capræ Saraceni acriter urticant, et palmis tundunt, ut in ea lac descendat. Postea lac in vas emulgitur, et in eo, per unam noctem, vitrum cum ferro ponitur, cum quo debet incidi; temperabitur in ipso lacte ferrum, aut in lotio parvæ puellæ rufæ, quod excipitur ante ortum solis. At vero lac, cum necesse fuerit, recalefiat eadem calitudine, qua fuit primitus mulsum, et in eo semper vitrum calefiat, donec molle fiat, et sic incidatur. Sic et aliæ petræ. Capra vero hedera pascatur.

\subsection{Ad cristallum comprimendum in figuram.}
\paragraph{}
Sume hircum qui nunquam coierit, et pone in cuppa per tres dies, quousque totum digerat quod in ventre habet. Postea hederam da ei edere per 4. dies; post hæc purgabis dolium, ut urinam ejus accipias. Post hæc occides hircum, et sanguinem ejus urinæ commiscebis; et sic lapidem impone per unam noctem, et post hæc, vel comprimes in figuram, vel sculpes si vis. Ut pulchrum facias, fac tibi tabulam plumbeam, et super hanc asperges album silicem contritum, ut piper, et lapidem desuper fricabis, quoadusque asperitatem lenies. Postea, liga de eodem silice contrito in laneo panno, et inde fricabis angulos quos prius aptare nequivisti in lamina. Deinde, ut pristinam lucem recipiat, fac tibi oleum de nucibus, et inde fricabis. Adhuc debes eum linire panno cerato, ut splendeat, et sudare desinat.

\subsection{Inauratio ferri.}
\paragraph{}
Eris limatura teritur cum aceto, in hereo mortario, cum sale et alumine, usque ad mellis spissitudinem. Aliqui pro aceto, aqua utuntur. Deinde ferrum bene purgatum, et leviter calefactum, hac mixtura inunguitur, et fricatur, donec colorem heris accipiat. Post hæc aqua abluitur, et tergitur, et sicut hes, vel argentum, deauratur; et calefactum, recedente vivo argento: sicut mos est, ut splendorem accipiat, ferro defricatur.

\subsection{Aliter.}
\paragraph{}
Alumen rotundum et salvandum, quod salis gemma vocatur, et calcantum, ex aceto acerrimo teruntur in hereo mortario. Ex his ferrum purgatum, cum ferura, vel alia qualibet levi hastula, defricatur; et, cum heris colorem habuerit, retergitur et deauratur, ac deinde, exfumigato vivo argento, aqua refrigeratur. Usque ad splendorem, ferro valde plano, et limpido, defricatur.

\subsection{De ebore.}
\paragraph{}
Quod si volueris ebur dirigere, vel carvare, in hac supradicta confectione mittatur tribus diebus et noctibus. Hoc facto, cavabis lignum quali modo volueris; deinde, posito ebore in cavatura, diriges illud, et plicabis ad placitum.

Accipe calcis vivæ duas partes, tegulæ tritæ unam partem, olei unam partem, stuppæ sissæ partem unam: distempera hæc omnia lesxivâ factâ de cortice hulmi.
\begin{center}
\textbf{Finis.}
\end{center}
\begin{center}
(\emph{On the back of this last leaf is written in nearly a cotemporary hand the following.})
\end{center}
\begin{table}[H]
    \centering
    \begin{tabular}{l l l}
        Cinnabarin  & 1 &  (\emph{i. e.}) Vermilio.                 \\
         Jarin       & 1 &  Flos eris.                        \\
         Psimithii   & 1 &  Flos plumbi.                      \\
         Magra       & 1 &  Sinopidum, vel Bolus Armeniacus. \\
    \end{tabular}
\end{table}
\end{document}
